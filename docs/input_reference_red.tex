\begin{RecordType}{\hyperB{IT::Root}{Root}}{}{}{Root record of JSON input for Flow123d.}
\KeyItem{\hyperB{Root::problem}{problem}}{abstract type: \Alink{IT::Problem}{Problem}}{\textlangle{\it obligatory }\textrangle}{}{Simulation problem to be solved.}
\KeyItem{\hyperB{Root::pause-after-run}{pause\_after\_run}}{String (generic)}{Bool}{\textlangle{\it value at declaration }\textrangle}{}{If true, the program will wait for key press before it terminates.}
\end{RecordType}

\begin{AbstractType}{\hyperB{IT::Problem}{Problem}}{}{}{The root record of description of particular the problem to solve.}
\Descendant{\Alink{IT::SequentialCoupling}{SequentialCoupling}}
\end{AbstractType}

\begin{RecordType}{\hyperB{IT::SequentialCoupling}{SequentialCoupling}}{\Alink{IT::Problem}{Problem}}{}{}{Record with data for a general sequential coupling.}
\KeyItem{\hyperB{SequentialCoupling::TYPE}{TYPE}}{selection: Problem\_TYPE\_selection}{SequentialCoupling}{}{Sub-record selection.}
\KeyItem{\hyperB{SequentialCoupling::description}{description}}{String (generic)}{\textlangle{\it optional }\textrangle}{}{Short description of the solved problem.\\Is displayed in the main log, and possibly in other text output files.}
\KeyItem{\hyperB{SequentialCoupling::mesh}{mesh}}{record: \Alink{IT::Mesh}{Mesh}}{\textlangle{\it obligatory }\textrangle}{}{Computational mesh common to all equations.}
\KeyItem{\hyperB{SequentialCoupling::time}{time}}{record: \Alink{IT::TimeGovernor}{TimeGovernor}}{\textlangle{\it optional }\textrangle}{}{Simulation time frame and time step.}
\KeyItem{\hyperB{SequentialCoupling::primary-equation}{primary\_equation}}{abstract type: \Alink{IT::DarcyFlowMH}{DarcyFlowMH}}{\textlangle{\it obligatory }\textrangle}{}{Primary equation, have all data given.}
\KeyItem{\hyperB{SequentialCoupling::secondary-equation}{secondary\_equation}}{abstract type: \Alink{IT::Transport}{Transport}}{\textlangle{\it optional }\textrangle}{}{The equation that depends (the velocity field) on the result of the primary equation.}
\end{RecordType}

\begin{SelectionType}{\hyperB{IT::Problem-TYPE-selection}{Problem\_TYPE\_selection}}{}
\KeyItem{SequentialCoupling}{}
\end{SelectionType}

\begin{RecordType}{\hyperB{IT::Mesh}{Mesh}}{}{}{Record with mesh related data.}
\KeyItem{\hyperB{Mesh::mesh-file}{mesh\_file}}{String (generic)}{input file name}{\textlangle{\it obligatory }\textrangle}{}{Input file with mesh description.}
\KeyItem{\hyperB{Mesh::regions}{regions}}{String (generic)}{Array of HOW_TO_DETERMINE_TYPE}{\textlangle{\it optional }\textrangle}{}{List of additional region definitions not contained in the mesh.}
\KeyItem{\hyperB{Mesh::sets}{sets}}{String (generic)}{Array of HOW_TO_DETERMINE_TYPE}{\textlangle{\it optional }\textrangle}{}{List of region set definitions. There are three region sets implicitly defined:\\ALL (all regions of the mesh), BOUNDARY (all boundary regions), and BULK (all bulk regions)}
\KeyItem{\hyperB{Mesh::partitioning}{partitioning}}{record: \Alink{IT::Partition}{Partition}}{\textlangle{\it value at declaration }\textrangle}{}{Parameters of mesh partitioning algorithms.}
\end{RecordType}

\begin{RecordType}{\hyperB{IT::Region}{Region}}{}{}{Definition of region of elements.}
\KeyItem{\hyperB{Region::name}{name}}{String (generic)}{\textlangle{\it obligatory }\textrangle}{}{Label (name) of the region. Has to be unique in one mesh.}
\KeyItem{\hyperB{Region::id}{id}}{String (generic)}{Integer<NumberRange {0 , 2147483647}>}{\textlangle{\it obligatory }\textrangle}{}{The ID of the region to which you assign label.}
\KeyItem{\hyperB{Region::element-list}{element\_list}}{String (generic)}{Array of HOW_TO_DETERMINE_TYPE}{\textlangle{\it optional }\textrangle}{}{Specification of the region by the list of elements. This is not recomended}
\end{RecordType}

\begin{RecordType}{\hyperB{IT::RegionSet}{RegionSet}}{}{}{Definition of one region set.}
\KeyItem{\hyperB{RegionSet::name}{name}}{String (generic)}{\textlangle{\it obligatory }\textrangle}{}{Unique name of the region set.}
\KeyItem{\hyperB{RegionSet::region-ids}{region\_ids}}{String (generic)}{Array of HOW_TO_DETERMINE_TYPE}{\textlangle{\it optional }\textrangle}{}{List of region ID numbers that has to be added to the region set.}
\KeyItem{\hyperB{RegionSet::region-labels}{region\_labels}}{String (generic)}{Array of HOW_TO_DETERMINE_TYPE}{\textlangle{\it optional }\textrangle}{}{List of labels of the regions that has to be added to the region set.}
\KeyItem{\hyperB{RegionSet::union}{union}}{String (generic)}{Array of HOW_TO_DETERMINE_TYPE}{\textlangle{\it optional }\textrangle}{}{Defines region set as a union of given pair of sets. Overrides previous keys.}
\KeyItem{\hyperB{RegionSet::intersection}{intersection}}{String (generic)}{Array of HOW_TO_DETERMINE_TYPE}{\textlangle{\it optional }\textrangle}{}{Defines region set as an intersection of given pair of sets. Overrides previous keys.}
\KeyItem{\hyperB{RegionSet::difference}{difference}}{String (generic)}{Array of HOW_TO_DETERMINE_TYPE}{\textlangle{\it optional }\textrangle}{}{Defines region set as a difference of given pair of sets. Overrides previous keys.}
\end{RecordType}

\begin{RecordType}{\hyperB{IT::Partition}{Partition}}{}{}{Setting for various types of mesh partitioning.}
\KeyItem{\hyperB{Partition::tool}{tool}}{selection: PartTool}{METIS}{}{Software package used for partitioning. See corresponding selection.}
\KeyItem{\hyperB{Partition::graph-type}{graph\_type}}{selection: GraphType}{any_neighboring}{}{Algorithm for generating graph and its weights from a multidimensional mesh.}
\end{RecordType}

\begin{SelectionType}{\hyperB{IT::PartTool}{PartTool}}{Select the partitioning tool to use.}
\KeyItem{PETSc}{Select the partitioning tool to use.}
\KeyItem{METIS}{Select the partitioning tool to use.}
\end{SelectionType}

\begin{SelectionType}{\hyperB{IT::GraphType}{GraphType}}{Different algorithms to make the sparse graph with weighted edges\\from the multidimensional mesh. Main difference is dealing with \\\eighborings of elements of different dimension.}
\KeyItem{any_neighboring}{Different algorithms to make the sparse graph with weighted edges
from the multidimensional mesh. Main difference is dealing with 
neighborings of elements of different dimension.}
\KeyItem{any_wight_lower_dim_cuts}{Different algorithms to make the sparse graph with weighted edges
from the multidimensional mesh. Main difference is dealing with 
neighborings of elements of different dimension.}
\KeyItem{same_dimension_neghboring}{Different algorithms to make the sparse graph with weighted edges
from the multidimensional mesh. Main difference is dealing with 
neighborings of elements of different dimension.}
\end{SelectionType}

\begin{RecordType}{\hyperB{IT::TimeGovernor}{TimeGovernor}}{}{}{Setting of the simulation time. (can be specific to one equation)}
\KeyItem{\hyperB{TimeGovernor::start-time}{start\_time}}{String (generic)}{Double<NumberRange {-1.79769e+308 , 1.79769e+308}>}{\textlangle{\it value at declaration }\textrangle}{}{Start time of the simulation.}
\KeyItem{\hyperB{TimeGovernor::end-time}{end\_time}}{String (generic)}{Double<NumberRange {-1.79769e+308 , 1.79769e+308}>}{\textlangle{\it value at read time }\textrangle}{}{End time of the simulation.}
\KeyItem{\hyperB{TimeGovernor::init-dt}{init\_dt}}{String (generic)}{Double<NumberRange {0 , 1.79769e+308}>}{\textlangle{\it value at declaration }\textrangle}{}{Initial guess for the time step.\\Only useful for equations that use adaptive time stepping.If set to 0.0, the time step is determined in fully autonomous way if the equation supports it.}
\KeyItem{\hyperB{TimeGovernor::min-dt}{min\_dt}}{String (generic)}{Double<NumberRange {0 , 1.79769e+308}>}{\textlangle{\it value at read time }\textrangle}{}{Soft lower limit for the time step. Equation using adaptive time stepping can notsuggest smaller time step, but actual time step could be smaller in order to match prescribed input or output times.}
\KeyItem{\hyperB{TimeGovernor::max-dt}{max\_dt}}{String (generic)}{Double<NumberRange {0 , 1.79769e+308}>}{\textlangle{\it value at read time }\textrangle}{}{Hard upper limit for the time step. Actual length of the time step is also limitedby input and output times.}
\end{RecordType}

\begin{AbstractType}{\hyperB{IT::DarcyFlowMH}{DarcyFlowMH}}{}{}{Mixed-Hybrid  solver for saturated Darcy flow.}
\Descendant{\Alink{IT::Steady-MH}{Steady\_MH}}
\Descendant{\Alink{IT::Unsteady-MH}{Unsteady\_MH}}
\Descendant{\Alink{IT::Unsteady-LMH}{Unsteady\_LMH}}
\end{AbstractType}

\begin{RecordType}{\hyperB{IT::Steady-MH}{Steady\_MH}}{\Alink{IT::DarcyFlowMH}{DarcyFlowMH}}{}{}{Mixed-Hybrid  solver for STEADY saturated Darcy flow.}
\KeyItem{\hyperB{Steady-MH::TYPE}{TYPE}}{selection: DarcyFlowMH\_TYPE\_selection}{Steady_MH}{}{Sub-record selection.}
\KeyItem{\hyperB{Steady-MH::n-schurs}{n\_schurs}}{String (generic)}{Integer<NumberRange {0 , 2}>}{\textlangle{\it value at declaration }\textrangle}{}{Number of Schur complements to perform when solving MH sytem.}
\KeyItem{\hyperB{Steady-MH::solver}{solver}}{abstract type: \Alink{IT::LinSys}{LinSys}}{\textlangle{\it obligatory }\textrangle}{}{Linear solver for MH problem.}
\KeyItem{\hyperB{Steady-MH::output}{output}}{record: \Alink{IT::DarcyMHOutput}{DarcyMHOutput}}{\textlangle{\it obligatory }\textrangle}{}{Parameters of output form MH module.}
\KeyItem{\hyperB{Steady-MH::mortar-method}{mortar\_method}}{selection: MH\_MortarMethod}{None}{}{Method for coupling Darcy flow between dimensions.}
\KeyItem{\hyperB{Steady-MH::balance}{balance}}{record: \Alink{IT::Balance}{Balance}}{\textlangle{\it obligatory }\textrangle}{}{Settings for computing mass balance.}
\KeyItem{\hyperB{Steady-MH::input-fields}{input\_fields}}{String (generic)}{Array of HOW_TO_DETERMINE_TYPE}{\textlangle{\it obligatory }\textrangle}{}{}
\end{RecordType}

\begin{SelectionType}{\hyperB{IT::DarcyFlowMH-TYPE-selection}{DarcyFlowMH\_TYPE\_selection}}{}
\KeyItem{Steady_MH}{}
\KeyItem{Unsteady_MH}{}
\KeyItem{Unsteady_LMH}{}
\end{SelectionType}

\begin{AbstractType}{\hyperB{IT::LinSys}{LinSys}}{}{}{Linear solver setting.}
\Descendant{\Alink{IT::Petsc}{Petsc}}
\Descendant{\Alink{IT::Bddc}{Bddc}}
\end{AbstractType}

\begin{RecordType}{\hyperB{IT::Petsc}{Petsc}}{\Alink{IT::LinSys}{LinSys}}{}{}{Solver setting.}
\KeyItem{\hyperB{Petsc::TYPE}{TYPE}}{selection: LinSys\_TYPE\_selection}{Petsc}{}{Sub-record selection.}
\KeyItem{\hyperB{Petsc::r-tol}{r\_tol}}{String (generic)}{Double<NumberRange {0 , 1}>}{\textlangle{\it value at declaration }\textrangle}{}{Relative residual tolerance (to initial error).}
\KeyItem{\hyperB{Petsc::max-it}{max\_it}}{String (generic)}{Integer<NumberRange {0 , 2147483647}>}{\textlangle{\it value at declaration }\textrangle}{}{Maximum number of outer iterations of the linear solver.}
\KeyItem{\hyperB{Petsc::a-tol}{a\_tol}}{String (generic)}{Double<NumberRange {0 , 1.79769e+308}>}{\textlangle{\it value at declaration }\textrangle}{}{Absolute residual tolerance.}
\KeyItem{\hyperB{Petsc::options}{options}}{String (generic)}{\textlangle{\it value at declaration }\textrangle}{}{Options passed to PETSC before creating KSP instead of default setting.}
\end{RecordType}

\begin{SelectionType}{\hyperB{IT::LinSys-TYPE-selection}{LinSys\_TYPE\_selection}}{}
\KeyItem{Petsc}{}
\KeyItem{Bddc}{}
\end{SelectionType}

\begin{RecordType}{\hyperB{IT::Bddc}{Bddc}}{\Alink{IT::LinSys}{LinSys}}{}{}{Solver setting.}
\KeyItem{\hyperB{Bddc::TYPE}{TYPE}}{selection: LinSys\_TYPE\_selection}{Bddc}{}{Sub-record selection.}
\KeyItem{\hyperB{Bddc::r-tol}{r\_tol}}{String (generic)}{Double<NumberRange {0 , 1}>}{\textlangle{\it value at declaration }\textrangle}{}{Relative residual tolerance (to initial error).}
\KeyItem{\hyperB{Bddc::max-it}{max\_it}}{String (generic)}{Integer<NumberRange {0 , 2147483647}>}{\textlangle{\it value at declaration }\textrangle}{}{Maximum number of outer iterations of the linear solver.}
\KeyItem{\hyperB{Bddc::max-nondecr-it}{max\_nondecr\_it}}{String (generic)}{Integer<NumberRange {0 , 2147483647}>}{\textlangle{\it value at declaration }\textrangle}{}{Maximum number of iterations of the linear solver with non-decreasing residual.}
\KeyItem{\hyperB{Bddc::number-of-levels}{number\_of\_levels}}{String (generic)}{Integer<NumberRange {0 , 2147483647}>}{\textlangle{\it value at declaration }\textrangle}{}{Number of levels in the multilevel method (=2 for the standard BDDC).}
\KeyItem{\hyperB{Bddc::use-adaptive-bddc}{use\_adaptive\_bddc}}{String (generic)}{Bool}{\textlangle{\it value at declaration }\textrangle}{}{Use adaptive selection of constraints in BDDCML.}
\KeyItem{\hyperB{Bddc::bddcml-verbosity-level}{bddcml\_verbosity\_level}}{String (generic)}{Integer<NumberRange {0 , 2}>}{\textlangle{\it value at declaration }\textrangle}{}{Level of verbosity of the BDDCML library: 0 - no output, 1 - mild output, 2 - detailed output.}
\end{RecordType}

\begin{RecordType}{\hyperB{IT::DarcyMHOutput}{DarcyMHOutput}}{}{}{Parameters of MH output.}
\KeyItem{\hyperB{DarcyMHOutput::output-stream}{output\_stream}}{record: \Alink{IT::OutputStream}{OutputStream}}{\textlangle{\it obligatory }\textrangle}{}{Parameters of output stream.}
\KeyItem{\hyperB{DarcyMHOutput::output-fields}{output\_fields}}{String (generic)}{Array of HOW_TO_DETERMINE_TYPE}{\textlangle{\it obligatory }\textrangle}{}{List of fields to write to output file.}
\KeyItem{\hyperB{DarcyMHOutput::compute-errors}{compute\_errors}}{String (generic)}{Bool}{\textlangle{\it value at declaration }\textrangle}{}{SPECIAL PURPOSE. Computing errors pro non-compatible coupling.}
\KeyItem{\hyperB{DarcyMHOutput::raw-flow-output}{raw\_flow\_output}}{String (generic)}{output file name}{\textlangle{\it optional }\textrangle}{}{Output file with raw data form MH module.}
\end{RecordType}

\begin{RecordType}{\hyperB{IT::OutputStream}{OutputStream}}{}{}{Parameters of output.}
\KeyItem{\hyperB{OutputStream::file}{file}}{String (generic)}{output file name}{\textlangle{\it obligatory }\textrangle}{}{File path to the connected output file.}
\KeyItem{\hyperB{OutputStream::format}{format}}{abstract type: \Alink{IT::OutputTime}{OutputTime}}{\textlangle{\it optional }\textrangle}{}{Format of output stream and possible parameters.}
\KeyItem{\hyperB{OutputStream::time-step}{time\_step}}{String (generic)}{Double<NumberRange {0 , 1.79769e+308}>}{\textlangle{\it optional }\textrangle}{}{Time interval between outputs.\\Regular grid of output time points starts at the initial time of the equation and ends at the end time which must be specified.\\The start time and the end time are always added.}
\KeyItem{\hyperB{OutputStream::time-list}{time\_list}}{String (generic)}{Array of HOW_TO_DETERMINE_TYPE}{\textlangle{\it value at read time }\textrangle}{}{Explicit array of output time points (can be combined with 'time_step'.}
\KeyItem{\hyperB{OutputStream::add-input-times}{add\_input\_times}}{String (generic)}{Bool}{\textlangle{\it value at declaration }\textrangle}{}{Add all input time points of the equation, mentioned in the 'input_fields' list, also as the output points.}
\end{RecordType}

\begin{AbstractType}{\hyperB{IT::OutputTime}{OutputTime}}{}{}{Format of output stream and possible parameters.}
\Descendant{\Alink{IT::vtk}{vtk}}
\Descendant{\Alink{IT::gmsh}{gmsh}}
\end{AbstractType}

\begin{RecordType}{\hyperB{IT::vtk}{vtk}}{\Alink{IT::OutputTime}{OutputTime}}{}{}{Parameters of vtk output format.}
\KeyItem{\hyperB{vtk::TYPE}{TYPE}}{selection: OutputTime\_TYPE\_selection}{vtk}{}{Sub-record selection.}
\KeyItem{\hyperB{vtk::variant}{variant}}{selection: VTK variant (ascii or binary)}{ascii}{}{Variant of output stream file format.}
\KeyItem{\hyperB{vtk::parallel}{parallel}}{String (generic)}{Bool}{\textlangle{\it value at declaration }\textrangle}{}{Parallel or serial version of file format.}
\KeyItem{\hyperB{vtk::compression}{compression}}{selection: Type of compression of VTK file format}{none}{}{Compression used in output stream file format.}
\end{RecordType}

\begin{SelectionType}{\hyperB{IT::OutputTime-TYPE-selection}{OutputTime\_TYPE\_selection}}{}
\KeyItem{vtk}{}
\KeyItem{gmsh}{}
\end{SelectionType}

\begin{SelectionType}{\hyperB{IT::VTK variant (ascii or binary)}{VTK variant (ascii or binary)}}{}
\KeyItem{ascii}{}
\KeyItem{binary}{}
\end{SelectionType}

\begin{SelectionType}{\hyperB{IT::Type of compression of VTK file format}{Type of compression of VTK file format}}{}
\KeyItem{none}{}
\KeyItem{zlib}{}
\end{SelectionType}

\begin{RecordType}{\hyperB{IT::gmsh}{gmsh}}{\Alink{IT::OutputTime}{OutputTime}}{}{}{Parameters of gmsh output format.}
\KeyItem{\hyperB{gmsh::TYPE}{TYPE}}{selection: OutputTime\_TYPE\_selection}{gmsh}{}{Sub-record selection.}
\end{RecordType}

\begin{SelectionType}{\hyperB{IT::DarcyMHOutput-Selection}{DarcyMHOutput\_Selection}}{Selection of fields available for output.}
\KeyItem{anisotropy}{Selection of fields available for output.}
\KeyItem{cross_section}{Selection of fields available for output.}
\KeyItem{conductivity}{Selection of fields available for output.}
\KeyItem{sigma}{Selection of fields available for output.}
\KeyItem{water_source_density}{Selection of fields available for output.}
\KeyItem{init_pressure}{Selection of fields available for output.}
\KeyItem{storativity}{Selection of fields available for output.}
\KeyItem{pressure_p0}{Selection of fields available for output.}
\KeyItem{pressure_p1}{Selection of fields available for output.}
\KeyItem{piezo_head_p0}{Selection of fields available for output.}
\KeyItem{velocity_p0}{Selection of fields available for output.}
\KeyItem{subdomain}{Selection of fields available for output.}
\KeyItem{region_id}{Selection of fields available for output.}
\KeyItem{pressure_diff}{Selection of fields available for output.}
\KeyItem{velocity_diff}{Selection of fields available for output.}
\KeyItem{div_diff}{Selection of fields available for output.}
\end{SelectionType}

\begin{SelectionType}{\hyperB{IT::MH-MortarMethod}{MH\_MortarMethod}}{}
\KeyItem{None}{}
\KeyItem{P0}{}
\KeyItem{P1}{}
\end{SelectionType}

\begin{RecordType}{\hyperB{IT::Balance}{Balance}}{}{}{Balance of a conservative quantity, boundary fluxes and sources.}
\KeyItem{\hyperB{Balance::balance-on}{balance\_on}}{String (generic)}{Bool}{\textlangle{\it value at declaration }\textrangle}{}{Balance is computed if the value is true.}
\KeyItem{\hyperB{Balance::format}{format}}{selection: Balance\_output\_format}{txt}{}{Format of output file.}
\KeyItem{\hyperB{Balance::cumulative}{cumulative}}{String (generic)}{Bool}{\textlangle{\it value at declaration }\textrangle}{}{Compute cumulative balance over time. If true, then balance is calculated at each computational time step, which can slow down the program.}
\KeyItem{\hyperB{Balance::file}{file}}{String (generic)}{output file name}{\textlangle{\it value at read time }\textrangle}{}{File name for output of balance.}
\end{RecordType}

\begin{SelectionType}{\hyperB{IT::Balance-output-format}{Balance\_output\_format}}{Format of output file for balance.}
\KeyItem{legacy}{Format of output file for balance.}
\KeyItem{txt}{Format of output file for balance.}
\KeyItem{gnuplot}{Format of output file for balance.}
\end{SelectionType}

\begin{RecordType}{\hyperB{IT::DarcyFlowMH-Data}{DarcyFlowMH\_Data}}{}{}{Record to set fields of the equation.\\The fields are set only on the domain specified by one of the keys: 'region', 'rid', 'r_set'\\and after the time given by the key 'time'. The field setting can be overridden by\\ any DarcyFlowMH_Data record that comes later in the boundary data array.}
\KeyItem{\hyperB{DarcyFlowMH-Data::r-set}{r\_set}}{String (generic)}{\textlangle{\it optional }\textrangle}{}{Name of region set where to set fields.}
\KeyItem{\hyperB{DarcyFlowMH-Data::region}{region}}{String (generic)}{\textlangle{\it optional }\textrangle}{}{Label of the region where to set fields.}
\KeyItem{\hyperB{DarcyFlowMH-Data::rid}{rid}}{String (generic)}{Integer<NumberRange {0 , 2147483647}>}{\textlangle{\it optional }\textrangle}{}{ID of the region where to set fields.}
\KeyItem{\hyperB{DarcyFlowMH-Data::time}{time}}{String (generic)}{Double<NumberRange {0 , 1.79769e+308}>}{\textlangle{\it value at declaration }\textrangle}{}{Apply field setting in this record after this time.\\These times have to form an increasing sequence.}
\KeyItem{\hyperB{DarcyFlowMH-Data::anisotropy}{anisotropy}}{abstract type: \Alink{IT::Field:R3 -> Real[3,3]}{Field:R3 -> Real[3,3]}}{\textlangle{\it optional }\textrangle}{}{Anisotropy of the conductivity tensor. $[-]$}
\KeyItem{\hyperB{DarcyFlowMH-Data::cross-section}{cross\_section}}{abstract type: \Alink{IT::Field:R3 -> Real}{Field:R3 -> Real}}{\textlangle{\it optional }\textrangle}{}{Complement dimension parameter (cross section for 1D, thickness for 2D). $[m^{3-d}]$}
\KeyItem{\hyperB{DarcyFlowMH-Data::conductivity}{conductivity}}{abstract type: \Alink{IT::Field:R3 -> Real}{Field:R3 -> Real}}{\textlangle{\it optional }\textrangle}{}{Isotropic conductivity scalar. $[ms^{-1}]$}
\KeyItem{\hyperB{DarcyFlowMH-Data::sigma}{sigma}}{abstract type: \Alink{IT::Field:R3 -> Real}{Field:R3 -> Real}}{\textlangle{\it optional }\textrangle}{}{Transition coefficient between dimensions. $[-]$}
\KeyItem{\hyperB{DarcyFlowMH-Data::water-source-density}{water\_source\_density}}{abstract type: \Alink{IT::Field:R3 -> Real}{Field:R3 -> Real}}{\textlangle{\it optional }\textrangle}{}{Water source density. $[s^{-1}]$}
\KeyItem{\hyperB{DarcyFlowMH-Data::bc-type}{bc\_type}}{abstract type: \Alink{IT::Field:R3 -> Enum}{Field:R3 -> Enum}}{\textlangle{\it optional }\textrangle}{}{Boundary condition type, possible values: $[-]$}
\KeyItem{\hyperB{DarcyFlowMH-Data::bc-pressure}{bc\_pressure}}{abstract type: \Alink{IT::Field:R3 -> Real}{Field:R3 -> Real}}{\textlangle{\it optional }\textrangle}{}{Dirichlet BC condition value for pressure. $[m]$}
\KeyItem{\hyperB{DarcyFlowMH-Data::bc-flux}{bc\_flux}}{abstract type: \Alink{IT::Field:R3 -> Real}{Field:R3 -> Real}}{\textlangle{\it optional }\textrangle}{}{Flux in Neumman or Robin boundary condition. $[m^{4-d}s^{-1}]$}
\KeyItem{\hyperB{DarcyFlowMH-Data::bc-robin-sigma}{bc\_robin\_sigma}}{abstract type: \Alink{IT::Field:R3 -> Real}{Field:R3 -> Real}}{\textlangle{\it optional }\textrangle}{}{Conductivity coefficient in Robin boundary condition. $[m^{3-d}s^{-1}]$}
\KeyItem{\hyperB{DarcyFlowMH-Data::init-pressure}{init\_pressure}}{abstract type: \Alink{IT::Field:R3 -> Real}{Field:R3 -> Real}}{\textlangle{\it optional }\textrangle}{}{Initial condition as pressure $[m]$}
\KeyItem{\hyperB{DarcyFlowMH-Data::storativity}{storativity}}{abstract type: \Alink{IT::Field:R3 -> Real}{Field:R3 -> Real}}{\textlangle{\it optional }\textrangle}{}{Storativity. $[m^{-1}]$}
\KeyItem{\hyperB{DarcyFlowMH-Data::bc-piezo-head}{bc\_piezo\_head}}{abstract type: \Alink{IT::Field:R3 -> Real}{Field:R3 -> Real}}{\textlangle{\it optional }\textrangle}{}{Boundary condition for pressure as piezometric head.}
\KeyItem{\hyperB{DarcyFlowMH-Data::init-piezo-head}{init\_piezo\_head}}{abstract type: \Alink{IT::Field:R3 -> Real}{Field:R3 -> Real}}{\textlangle{\it optional }\textrangle}{}{Initial condition for pressure as piezometric head.}
\KeyItem{\hyperB{DarcyFlowMH-Data::flow-old-bcd-file}{flow\_old\_bcd\_file}}{String (generic)}{input file name}{\textlangle{\it optional }\textrangle}{}{File with mesh dependent boundary conditions (obsolete).}
\end{RecordType}

\begin{AbstractType}{\hyperB{IT::Field:R3 -> Real[3,3]}{Field:R3 -> Real[3,3]}}{}{}{Abstract record for all time-space functions.}
\Descendant{\Alink{IT::FieldConstant}{FieldConstant}}
\Descendant{\Alink{IT::FieldPython}{FieldPython}}
\Descendant{\Alink{IT::FieldFormula}{FieldFormula}}
\Descendant{\Alink{IT::FieldElementwise}{FieldElementwise}}
\Descendant{\Alink{IT::FieldInterpolatedP0}{FieldInterpolatedP0}}
\end{AbstractType}

\begin{RecordType}{\hyperB{IT::FieldConstant}{FieldConstant}}{\Alink{IT::Field:R3 -> Real[3,3]}{Field:R3 -> Real[3,3]}}{}{}{R3 -> Real[3,3] Field constant in space.}
\KeyItem{\hyperB{FieldConstant::TYPE}{TYPE}}{selection: Field:R3 -> Real[3,3]\_TYPE\_selection}{FieldConstant}{}{Sub-record selection.}
\KeyItem{\hyperB{FieldConstant::value}{value}}{String (generic)}{Array of HOW_TO_DETERMINE_TYPE}{\textlangle{\it obligatory }\textrangle}{}{Value of the constant field.\\For vector values, you can use scalar value to enter constant vector.\\For square NxN-matrix values, you can use:\\* vector of size N to enter diagonal matrix\\* vector of size (N+1)*N/2 to enter symmetric matrix (upper triangle, row by row)\\* scalar to enter multiple of the unit matrix.}
\end{RecordType}

\begin{SelectionType}{\hyperB{IT::Field:R3 -> Real[3,3]-TYPE-selection}{Field:R3 -> Real[3,3]\_TYPE\_selection}}{}
\KeyItem{FieldConstant}{}
\KeyItem{FieldPython}{}
\KeyItem{FieldFormula}{}
\KeyItem{FieldElementwise}{}
\KeyItem{FieldInterpolatedP0}{}
\end{SelectionType}

\begin{RecordType}{\hyperB{IT::FieldPython}{FieldPython}}{\Alink{IT::Field:R3 -> Real[3,3]}{Field:R3 -> Real[3,3]}}{}{}{R3 -> Real[3,3] Field given by a Python script.}
\KeyItem{\hyperB{FieldPython::TYPE}{TYPE}}{selection: Field:R3 -> Real[3,3]\_TYPE\_selection}{FieldPython}{}{Sub-record selection.}
\KeyItem{\hyperB{FieldPython::script-string}{script\_string}}{String (generic)}{\textlangle{\it value at read time }\textrangle}{}{Python script given as in place string}
\KeyItem{\hyperB{FieldPython::script-file}{script\_file}}{String (generic)}{input file name}{\textlangle{\it value at read time }\textrangle}{}{Python script given as external file}
\KeyItem{\hyperB{FieldPython::function}{function}}{String (generic)}{\textlangle{\it obligatory }\textrangle}{}{Function in the given script that returns tuple containing components of the return type.\\For NxM tensor values: tensor(row,col) = tuple( M*row + col ).}
\end{RecordType}

\begin{RecordType}{\hyperB{IT::FieldFormula}{FieldFormula}}{\Alink{IT::Field:R3 -> Real[3,3]}{Field:R3 -> Real[3,3]}}{}{}{R3 -> Real[3,3] Field given by runtime interpreted formula.}
\KeyItem{\hyperB{FieldFormula::TYPE}{TYPE}}{selection: Field:R3 -> Real[3,3]\_TYPE\_selection}{FieldFormula}{}{Sub-record selection.}
\KeyItem{\hyperB{FieldFormula::value}{value}}{String (generic)}{Array of HOW_TO_DETERMINE_TYPE}{\textlangle{\it obligatory }\textrangle}{}{String, array of strings, or matrix of strings with formulas for individual entries of scalar, vector, or tensor value respectively.\\For vector values, you can use just one string to enter homogeneous vector.\\For square NxN-matrix values, you can use:\\* array of strings of size N to enter diagonal matrix\\* array of strings of size (N+1)*N/2 to enter symmetric matrix (upper triangle, row by row)\\* just one string to enter (spatially variable) multiple of the unit matrix.\\Formula can contain variables x,y,z,t and usual operators and functions.}
\end{RecordType}

\begin{RecordType}{\hyperB{IT::FieldElementwise}{FieldElementwise}}{\Alink{IT::Field:R3 -> Real[3,3]}{Field:R3 -> Real[3,3]}}{}{}{R3 -> Real[3,3] Field constant in space.}
\KeyItem{\hyperB{FieldElementwise::TYPE}{TYPE}}{selection: Field:R3 -> Real[3,3]\_TYPE\_selection}{FieldElementwise}{}{Sub-record selection.}
\KeyItem{\hyperB{FieldElementwise::gmsh-file}{gmsh\_file}}{String (generic)}{input file name}{\textlangle{\it obligatory }\textrangle}{}{Input file with ASCII GMSH file format.}
\KeyItem{\hyperB{FieldElementwise::field-name}{field\_name}}{String (generic)}{\textlangle{\it obligatory }\textrangle}{}{The values of the Field are read from the $ElementData section with field name given by this key.}
\end{RecordType}

\begin{RecordType}{\hyperB{IT::FieldInterpolatedP0}{FieldInterpolatedP0}}{\Alink{IT::Field:R3 -> Real[3,3]}{Field:R3 -> Real[3,3]}}{}{}{R3 -> Real[3,3] Field constant in space.}
\KeyItem{\hyperB{FieldInterpolatedP0::TYPE}{TYPE}}{selection: Field:R3 -> Real[3,3]\_TYPE\_selection}{FieldInterpolatedP0}{}{Sub-record selection.}
\KeyItem{\hyperB{FieldInterpolatedP0::gmsh-file}{gmsh\_file}}{String (generic)}{input file name}{\textlangle{\it obligatory }\textrangle}{}{Input file with ASCII GMSH file format.}
\KeyItem{\hyperB{FieldInterpolatedP0::field-name}{field\_name}}{String (generic)}{\textlangle{\it obligatory }\textrangle}{}{The values of the Field are read from the $ElementData section with field name given by this key.}
\end{RecordType}

\begin{AbstractType}{\hyperB{IT::Field:R3 -> Real}{Field:R3 -> Real}}{}{}{Abstract record for all time-space functions.}
\Descendant{\Alink{IT::FieldConstant}{FieldConstant}}
\Descendant{\Alink{IT::FieldPython}{FieldPython}}
\Descendant{\Alink{IT::FieldFormula}{FieldFormula}}
\Descendant{\Alink{IT::FieldElementwise}{FieldElementwise}}
\Descendant{\Alink{IT::FieldInterpolatedP0}{FieldInterpolatedP0}}
\end{AbstractType}

\begin{RecordType}{\hyperB{IT::FieldConstant}{FieldConstant}}{\Alink{IT::Field:R3 -> Real}{Field:R3 -> Real}}{}{}{R3 -> Real Field constant in space.}
\KeyItem{\hyperB{FieldConstant::TYPE}{TYPE}}{selection: Field:R3 -> Real\_TYPE\_selection}{FieldConstant}{}{Sub-record selection.}
\KeyItem{\hyperB{FieldConstant::value}{value}}{String (generic)}{Double<NumberRange {-1.79769e+308 , 1.79769e+308}>}{\textlangle{\it obligatory }\textrangle}{}{Value of the constant field.\\For vector values, you can use scalar value to enter constant vector.\\For square NxN-matrix values, you can use:\\* vector of size N to enter diagonal matrix\\* vector of size (N+1)*N/2 to enter symmetric matrix (upper triangle, row by row)\\* scalar to enter multiple of the unit matrix.}
\end{RecordType}

\begin{SelectionType}{\hyperB{IT::Field:R3 -> Real-TYPE-selection}{Field:R3 -> Real\_TYPE\_selection}}{}
\KeyItem{FieldConstant}{}
\KeyItem{FieldPython}{}
\KeyItem{FieldFormula}{}
\KeyItem{FieldElementwise}{}
\KeyItem{FieldInterpolatedP0}{}
\end{SelectionType}

\begin{RecordType}{\hyperB{IT::FieldPython}{FieldPython}}{\Alink{IT::Field:R3 -> Real}{Field:R3 -> Real}}{}{}{R3 -> Real Field given by a Python script.}
\KeyItem{\hyperB{FieldPython::TYPE}{TYPE}}{selection: Field:R3 -> Real\_TYPE\_selection}{FieldPython}{}{Sub-record selection.}
\KeyItem{\hyperB{FieldPython::script-string}{script\_string}}{String (generic)}{\textlangle{\it value at read time }\textrangle}{}{Python script given as in place string}
\KeyItem{\hyperB{FieldPython::script-file}{script\_file}}{String (generic)}{input file name}{\textlangle{\it value at read time }\textrangle}{}{Python script given as external file}
\KeyItem{\hyperB{FieldPython::function}{function}}{String (generic)}{\textlangle{\it obligatory }\textrangle}{}{Function in the given script that returns tuple containing components of the return type.\\For NxM tensor values: tensor(row,col) = tuple( M*row + col ).}
\end{RecordType}

\begin{RecordType}{\hyperB{IT::FieldFormula}{FieldFormula}}{\Alink{IT::Field:R3 -> Real}{Field:R3 -> Real}}{}{}{R3 -> Real Field given by runtime interpreted formula.}
\KeyItem{\hyperB{FieldFormula::TYPE}{TYPE}}{selection: Field:R3 -> Real\_TYPE\_selection}{FieldFormula}{}{Sub-record selection.}
\KeyItem{\hyperB{FieldFormula::value}{value}}{String (generic)}{\textlangle{\it obligatory }\textrangle}{}{String, array of strings, or matrix of strings with formulas for individual entries of scalar, vector, or tensor value respectively.\\For vector values, you can use just one string to enter homogeneous vector.\\For square NxN-matrix values, you can use:\\* array of strings of size N to enter diagonal matrix\\* array of strings of size (N+1)*N/2 to enter symmetric matrix (upper triangle, row by row)\\* just one string to enter (spatially variable) multiple of the unit matrix.\\Formula can contain variables x,y,z,t and usual operators and functions.}
\end{RecordType}

\begin{RecordType}{\hyperB{IT::FieldElementwise}{FieldElementwise}}{\Alink{IT::Field:R3 -> Real}{Field:R3 -> Real}}{}{}{R3 -> Real Field constant in space.}
\KeyItem{\hyperB{FieldElementwise::TYPE}{TYPE}}{selection: Field:R3 -> Real\_TYPE\_selection}{FieldElementwise}{}{Sub-record selection.}
\KeyItem{\hyperB{FieldElementwise::gmsh-file}{gmsh\_file}}{String (generic)}{input file name}{\textlangle{\it obligatory }\textrangle}{}{Input file with ASCII GMSH file format.}
\KeyItem{\hyperB{FieldElementwise::field-name}{field\_name}}{String (generic)}{\textlangle{\it obligatory }\textrangle}{}{The values of the Field are read from the $ElementData section with field name given by this key.}
\end{RecordType}

\begin{RecordType}{\hyperB{IT::FieldInterpolatedP0}{FieldInterpolatedP0}}{\Alink{IT::Field:R3 -> Real}{Field:R3 -> Real}}{}{}{R3 -> Real Field constant in space.}
\KeyItem{\hyperB{FieldInterpolatedP0::TYPE}{TYPE}}{selection: Field:R3 -> Real\_TYPE\_selection}{FieldInterpolatedP0}{}{Sub-record selection.}
\KeyItem{\hyperB{FieldInterpolatedP0::gmsh-file}{gmsh\_file}}{String (generic)}{input file name}{\textlangle{\it obligatory }\textrangle}{}{Input file with ASCII GMSH file format.}
\KeyItem{\hyperB{FieldInterpolatedP0::field-name}{field\_name}}{String (generic)}{\textlangle{\it obligatory }\textrangle}{}{The values of the Field are read from the $ElementData section with field name given by this key.}
\end{RecordType}

\begin{AbstractType}{\hyperB{IT::Field:R3 -> Enum}{Field:R3 -> Enum}}{}{}{Abstract record for all time-space functions.}
\Descendant{\Alink{IT::FieldConstant}{FieldConstant}}
\Descendant{\Alink{IT::FieldFormula}{FieldFormula}}
\Descendant{\Alink{IT::FieldPython}{FieldPython}}
\Descendant{\Alink{IT::FieldInterpolatedP0}{FieldInterpolatedP0}}
\Descendant{\Alink{IT::FieldElementwise}{FieldElementwise}}
\end{AbstractType}

\begin{RecordType}{\hyperB{IT::FieldConstant}{FieldConstant}}{\Alink{IT::Field:R3 -> Enum}{Field:R3 -> Enum}}{}{}{R3 -> Enum Field constant in space.}
\KeyItem{\hyperB{FieldConstant::TYPE}{TYPE}}{selection: Field:R3 -> Enum\_TYPE\_selection}{FieldConstant}{}{Sub-record selection.}
\KeyItem{\hyperB{FieldConstant::value}{value}}{selection: DarcyFlow\_BC\_Type}{OBLIGATORY}{}{Value of the constant field.\\For vector values, you can use scalar value to enter constant vector.\\For square NxN-matrix values, you can use:\\* vector of size N to enter diagonal matrix\\* vector of size (N+1)*N/2 to enter symmetric matrix (upper triangle, row by row)\\* scalar to enter multiple of the unit matrix.}
\end{RecordType}

\begin{SelectionType}{\hyperB{IT::Field:R3 -> Enum-TYPE-selection}{Field:R3 -> Enum\_TYPE\_selection}}{}
\KeyItem{FieldConstant}{}
\KeyItem{FieldFormula}{}
\KeyItem{FieldPython}{}
\KeyItem{FieldInterpolatedP0}{}
\KeyItem{FieldElementwise}{}
\end{SelectionType}

\begin{SelectionType}{\hyperB{IT::DarcyFlow-BC-Type}{DarcyFlow\_BC\_Type}}{}
\KeyItem{none}{}
\KeyItem{dirichlet}{}
\KeyItem{neumann}{}
\KeyItem{robin}{}
\end{SelectionType}

\begin{RecordType}{\hyperB{IT::FieldFormula}{FieldFormula}}{\Alink{IT::Field:R3 -> Enum}{Field:R3 -> Enum}}{}{}{R3 -> Enum Field given by runtime interpreted formula.}
\KeyItem{\hyperB{FieldFormula::TYPE}{TYPE}}{selection: Field:R3 -> Enum\_TYPE\_selection}{FieldFormula}{}{Sub-record selection.}
\KeyItem{\hyperB{FieldFormula::value}{value}}{String (generic)}{\textlangle{\it obligatory }\textrangle}{}{String, array of strings, or matrix of strings with formulas for individual entries of scalar, vector, or tensor value respectively.\\For vector values, you can use just one string to enter homogeneous vector.\\For square NxN-matrix values, you can use:\\* array of strings of size N to enter diagonal matrix\\* array of strings of size (N+1)*N/2 to enter symmetric matrix (upper triangle, row by row)\\* just one string to enter (spatially variable) multiple of the unit matrix.\\Formula can contain variables x,y,z,t and usual operators and functions.}
\end{RecordType}

\begin{RecordType}{\hyperB{IT::FieldPython}{FieldPython}}{\Alink{IT::Field:R3 -> Enum}{Field:R3 -> Enum}}{}{}{R3 -> Enum Field given by a Python script.}
\KeyItem{\hyperB{FieldPython::TYPE}{TYPE}}{selection: Field:R3 -> Enum\_TYPE\_selection}{FieldPython}{}{Sub-record selection.}
\KeyItem{\hyperB{FieldPython::script-string}{script\_string}}{String (generic)}{\textlangle{\it value at read time }\textrangle}{}{Python script given as in place string}
\KeyItem{\hyperB{FieldPython::script-file}{script\_file}}{String (generic)}{input file name}{\textlangle{\it value at read time }\textrangle}{}{Python script given as external file}
\KeyItem{\hyperB{FieldPython::function}{function}}{String (generic)}{\textlangle{\it obligatory }\textrangle}{}{Function in the given script that returns tuple containing components of the return type.\\For NxM tensor values: tensor(row,col) = tuple( M*row + col ).}
\end{RecordType}

\begin{RecordType}{\hyperB{IT::FieldInterpolatedP0}{FieldInterpolatedP0}}{\Alink{IT::Field:R3 -> Enum}{Field:R3 -> Enum}}{}{}{R3 -> Enum Field constant in space.}
\KeyItem{\hyperB{FieldInterpolatedP0::TYPE}{TYPE}}{selection: Field:R3 -> Enum\_TYPE\_selection}{FieldInterpolatedP0}{}{Sub-record selection.}
\KeyItem{\hyperB{FieldInterpolatedP0::gmsh-file}{gmsh\_file}}{String (generic)}{input file name}{\textlangle{\it obligatory }\textrangle}{}{Input file with ASCII GMSH file format.}
\KeyItem{\hyperB{FieldInterpolatedP0::field-name}{field\_name}}{String (generic)}{\textlangle{\it obligatory }\textrangle}{}{The values of the Field are read from the $ElementData section with field name given by this key.}
\end{RecordType}

\begin{RecordType}{\hyperB{IT::FieldElementwise}{FieldElementwise}}{\Alink{IT::Field:R3 -> Enum}{Field:R3 -> Enum}}{}{}{R3 -> Enum Field constant in space.}
\KeyItem{\hyperB{FieldElementwise::TYPE}{TYPE}}{selection: Field:R3 -> Enum\_TYPE\_selection}{FieldElementwise}{}{Sub-record selection.}
\KeyItem{\hyperB{FieldElementwise::gmsh-file}{gmsh\_file}}{String (generic)}{input file name}{\textlangle{\it obligatory }\textrangle}{}{Input file with ASCII GMSH file format.}
\KeyItem{\hyperB{FieldElementwise::field-name}{field\_name}}{String (generic)}{\textlangle{\it obligatory }\textrangle}{}{The values of the Field are read from the $ElementData section with field name given by this key.}
\end{RecordType}

\begin{AbstractType}{\hyperB{IT::Field:R3 -> Real}{Field:R3 -> Real}}{}{}{Abstract record for all time-space functions.}
\Descendant{\Alink{IT::FieldConstant}{FieldConstant}}
\Descendant{\Alink{IT::FieldFormula}{FieldFormula}}
\Descendant{\Alink{IT::FieldPython}{FieldPython}}
\Descendant{\Alink{IT::FieldInterpolatedP0}{FieldInterpolatedP0}}
\Descendant{\Alink{IT::FieldElementwise}{FieldElementwise}}
\end{AbstractType}

\begin{RecordType}{\hyperB{IT::FieldConstant}{FieldConstant}}{\Alink{IT::Field:R3 -> Real}{Field:R3 -> Real}}{}{}{R3 -> Real Field constant in space.}
\KeyItem{\hyperB{FieldConstant::TYPE}{TYPE}}{selection: Field:R3 -> Real\_TYPE\_selection}{FieldConstant}{}{Sub-record selection.}
\KeyItem{\hyperB{FieldConstant::value}{value}}{String (generic)}{Double<NumberRange {-1.79769e+308 , 1.79769e+308}>}{\textlangle{\it obligatory }\textrangle}{}{Value of the constant field.\\For vector values, you can use scalar value to enter constant vector.\\For square NxN-matrix values, you can use:\\* vector of size N to enter diagonal matrix\\* vector of size (N+1)*N/2 to enter symmetric matrix (upper triangle, row by row)\\* scalar to enter multiple of the unit matrix.}
\end{RecordType}

\begin{SelectionType}{\hyperB{IT::Field:R3 -> Real-TYPE-selection}{Field:R3 -> Real\_TYPE\_selection}}{}
\KeyItem{FieldConstant}{}
\KeyItem{FieldFormula}{}
\KeyItem{FieldPython}{}
\KeyItem{FieldInterpolatedP0}{}
\KeyItem{FieldElementwise}{}
\end{SelectionType}

\begin{RecordType}{\hyperB{IT::FieldFormula}{FieldFormula}}{\Alink{IT::Field:R3 -> Real}{Field:R3 -> Real}}{}{}{R3 -> Real Field given by runtime interpreted formula.}
\KeyItem{\hyperB{FieldFormula::TYPE}{TYPE}}{selection: Field:R3 -> Real\_TYPE\_selection}{FieldFormula}{}{Sub-record selection.}
\KeyItem{\hyperB{FieldFormula::value}{value}}{String (generic)}{\textlangle{\it obligatory }\textrangle}{}{String, array of strings, or matrix of strings with formulas for individual entries of scalar, vector, or tensor value respectively.\\For vector values, you can use just one string to enter homogeneous vector.\\For square NxN-matrix values, you can use:\\* array of strings of size N to enter diagonal matrix\\* array of strings of size (N+1)*N/2 to enter symmetric matrix (upper triangle, row by row)\\* just one string to enter (spatially variable) multiple of the unit matrix.\\Formula can contain variables x,y,z,t and usual operators and functions.}
\end{RecordType}

\begin{RecordType}{\hyperB{IT::FieldPython}{FieldPython}}{\Alink{IT::Field:R3 -> Real}{Field:R3 -> Real}}{}{}{R3 -> Real Field given by a Python script.}
\KeyItem{\hyperB{FieldPython::TYPE}{TYPE}}{selection: Field:R3 -> Real\_TYPE\_selection}{FieldPython}{}{Sub-record selection.}
\KeyItem{\hyperB{FieldPython::script-string}{script\_string}}{String (generic)}{\textlangle{\it value at read time }\textrangle}{}{Python script given as in place string}
\KeyItem{\hyperB{FieldPython::script-file}{script\_file}}{String (generic)}{input file name}{\textlangle{\it value at read time }\textrangle}{}{Python script given as external file}
\KeyItem{\hyperB{FieldPython::function}{function}}{String (generic)}{\textlangle{\it obligatory }\textrangle}{}{Function in the given script that returns tuple containing components of the return type.\\For NxM tensor values: tensor(row,col) = tuple( M*row + col ).}
\end{RecordType}

\begin{RecordType}{\hyperB{IT::FieldInterpolatedP0}{FieldInterpolatedP0}}{\Alink{IT::Field:R3 -> Real}{Field:R3 -> Real}}{}{}{R3 -> Real Field constant in space.}
\KeyItem{\hyperB{FieldInterpolatedP0::TYPE}{TYPE}}{selection: Field:R3 -> Real\_TYPE\_selection}{FieldInterpolatedP0}{}{Sub-record selection.}
\KeyItem{\hyperB{FieldInterpolatedP0::gmsh-file}{gmsh\_file}}{String (generic)}{input file name}{\textlangle{\it obligatory }\textrangle}{}{Input file with ASCII GMSH file format.}
\KeyItem{\hyperB{FieldInterpolatedP0::field-name}{field\_name}}{String (generic)}{\textlangle{\it obligatory }\textrangle}{}{The values of the Field are read from the $ElementData section with field name given by this key.}
\end{RecordType}

\begin{RecordType}{\hyperB{IT::FieldElementwise}{FieldElementwise}}{\Alink{IT::Field:R3 -> Real}{Field:R3 -> Real}}{}{}{R3 -> Real Field constant in space.}
\KeyItem{\hyperB{FieldElementwise::TYPE}{TYPE}}{selection: Field:R3 -> Real\_TYPE\_selection}{FieldElementwise}{}{Sub-record selection.}
\KeyItem{\hyperB{FieldElementwise::gmsh-file}{gmsh\_file}}{String (generic)}{input file name}{\textlangle{\it obligatory }\textrangle}{}{Input file with ASCII GMSH file format.}
\KeyItem{\hyperB{FieldElementwise::field-name}{field\_name}}{String (generic)}{\textlangle{\it obligatory }\textrangle}{}{The values of the Field are read from the $ElementData section with field name given by this key.}
\end{RecordType}

\begin{AbstractType}{\hyperB{IT::Field:R3 -> Real}{Field:R3 -> Real}}{}{}{Abstract record for all time-space functions.}
\Descendant{\Alink{IT::FieldConstant}{FieldConstant}}
\Descendant{\Alink{IT::FieldFormula}{FieldFormula}}
\Descendant{\Alink{IT::FieldPython}{FieldPython}}
\Descendant{\Alink{IT::FieldInterpolatedP0}{FieldInterpolatedP0}}
\Descendant{\Alink{IT::FieldElementwise}{FieldElementwise}}
\end{AbstractType}

\begin{RecordType}{\hyperB{IT::Unsteady-MH}{Unsteady\_MH}}{\Alink{IT::DarcyFlowMH}{DarcyFlowMH}}{}{}{Mixed-Hybrid solver for unsteady saturated Darcy flow.}
\KeyItem{\hyperB{Unsteady-MH::TYPE}{TYPE}}{selection: DarcyFlowMH\_TYPE\_selection}{Unsteady_MH}{}{Sub-record selection.}
\KeyItem{\hyperB{Unsteady-MH::n-schurs}{n\_schurs}}{String (generic)}{Integer<NumberRange {0 , 2}>}{\textlangle{\it value at declaration }\textrangle}{}{Number of Schur complements to perform when solving MH sytem.}
\KeyItem{\hyperB{Unsteady-MH::solver}{solver}}{abstract type: \Alink{IT::LinSys}{LinSys}}{\textlangle{\it obligatory }\textrangle}{}{Linear solver for MH problem.}
\KeyItem{\hyperB{Unsteady-MH::output}{output}}{record: \Alink{IT::DarcyMHOutput}{DarcyMHOutput}}{\textlangle{\it obligatory }\textrangle}{}{Parameters of output form MH module.}
\KeyItem{\hyperB{Unsteady-MH::mortar-method}{mortar\_method}}{selection: MH\_MortarMethod}{None}{}{Method for coupling Darcy flow between dimensions.}
\KeyItem{\hyperB{Unsteady-MH::balance}{balance}}{record: \Alink{IT::Balance}{Balance}}{\textlangle{\it obligatory }\textrangle}{}{Settings for computing mass balance.}
\KeyItem{\hyperB{Unsteady-MH::input-fields}{input\_fields}}{String (generic)}{Array of HOW_TO_DETERMINE_TYPE}{\textlangle{\it obligatory }\textrangle}{}{}
\KeyItem{\hyperB{Unsteady-MH::time}{time}}{record: \Alink{IT::TimeGovernor}{TimeGovernor}}{\textlangle{\it obligatory }\textrangle}{}{Time governor setting for the unsteady Darcy flow model.}
\end{RecordType}

\begin{RecordType}{\hyperB{IT::Unsteady-LMH}{Unsteady\_LMH}}{\Alink{IT::DarcyFlowMH}{DarcyFlowMH}}{}{}{Lumped Mixed-Hybrid solver for unsteady saturated Darcy flow.}
\KeyItem{\hyperB{Unsteady-LMH::TYPE}{TYPE}}{selection: DarcyFlowMH\_TYPE\_selection}{Unsteady_LMH}{}{Sub-record selection.}
\KeyItem{\hyperB{Unsteady-LMH::n-schurs}{n\_schurs}}{String (generic)}{Integer<NumberRange {0 , 2}>}{\textlangle{\it value at declaration }\textrangle}{}{Number of Schur complements to perform when solving MH sytem.}
\KeyItem{\hyperB{Unsteady-LMH::solver}{solver}}{abstract type: \Alink{IT::LinSys}{LinSys}}{\textlangle{\it obligatory }\textrangle}{}{Linear solver for MH problem.}
\KeyItem{\hyperB{Unsteady-LMH::output}{output}}{record: \Alink{IT::DarcyMHOutput}{DarcyMHOutput}}{\textlangle{\it obligatory }\textrangle}{}{Parameters of output form MH module.}
\KeyItem{\hyperB{Unsteady-LMH::mortar-method}{mortar\_method}}{selection: MH\_MortarMethod}{None}{}{Method for coupling Darcy flow between dimensions.}
\KeyItem{\hyperB{Unsteady-LMH::balance}{balance}}{record: \Alink{IT::Balance}{Balance}}{\textlangle{\it obligatory }\textrangle}{}{Settings for computing mass balance.}
\KeyItem{\hyperB{Unsteady-LMH::input-fields}{input\_fields}}{String (generic)}{Array of HOW_TO_DETERMINE_TYPE}{\textlangle{\it obligatory }\textrangle}{}{}
\KeyItem{\hyperB{Unsteady-LMH::time}{time}}{record: \Alink{IT::TimeGovernor}{TimeGovernor}}{\textlangle{\it obligatory }\textrangle}{}{Time governor setting for the unsteady Darcy flow model.}
\end{RecordType}

\begin{AbstractType}{\hyperB{IT::Transport}{Transport}}{}{}{Secondary equation for transport of substances.}
\Descendant{\Alink{IT::TransportOperatorSplitting}{TransportOperatorSplitting}}
\Descendant{\Alink{IT::SoluteTransport-DG}{SoluteTransport\_DG}}
\Descendant{\Alink{IT::HeatTransfer-DG}{HeatTransfer\_DG}}
\end{AbstractType}

\begin{RecordType}{\hyperB{IT::TransportOperatorSplitting}{TransportOperatorSplitting}}{\Alink{IT::Transport}{Transport}}{}{}{Explicit FVM transport (no diffusion)\\coupled with reaction and adsorption model (ODE per element)\\ via operator splitting.}
\KeyItem{\hyperB{TransportOperatorSplitting::TYPE}{TYPE}}{selection: Transport\_TYPE\_selection}{TransportOperatorSplitting}{}{Sub-record selection.}
\KeyItem{\hyperB{TransportOperatorSplitting::time}{time}}{record: \Alink{IT::TimeGovernor}{TimeGovernor}}{\textlangle{\it obligatory }\textrangle}{}{Time governor setting for the secondary equation.}
\KeyItem{\hyperB{TransportOperatorSplitting::balance}{balance}}{record: \Alink{IT::Balance}{Balance}}{\textlangle{\it obligatory }\textrangle}{}{Settings for computing balance.}
\KeyItem{\hyperB{TransportOperatorSplitting::output-stream}{output\_stream}}{record: \Alink{IT::OutputStream}{OutputStream}}{\textlangle{\it obligatory }\textrangle}{}{Parameters of output stream.}
\KeyItem{\hyperB{TransportOperatorSplitting::substances}{substances}}{String (generic)}{Array of HOW_TO_DETERMINE_TYPE}{\textlangle{\it obligatory }\textrangle}{}{Specification of transported substances.}
\KeyItem{\hyperB{TransportOperatorSplitting::reaction-term}{reaction\_term}}{abstract type: \Alink{IT::ReactionTerm}{ReactionTerm}}{\textlangle{\it optional }\textrangle}{}{Reaction model involved in transport.}
\KeyItem{\hyperB{TransportOperatorSplitting::input-fields}{input\_fields}}{String (generic)}{Array of HOW_TO_DETERMINE_TYPE}{\textlangle{\it obligatory }\textrangle}{}{}
\KeyItem{\hyperB{TransportOperatorSplitting::output-fields}{output\_fields}}{String (generic)}{Array of HOW_TO_DETERMINE_TYPE}{\textlangle{\it value at declaration }\textrangle}{}{List of fields to write to output file.}
\end{RecordType}

\begin{SelectionType}{\hyperB{IT::Transport-TYPE-selection}{Transport\_TYPE\_selection}}{}
\KeyItem{TransportOperatorSplitting}{}
\KeyItem{SoluteTransport_DG}{}
\KeyItem{HeatTransfer_DG}{}
\end{SelectionType}

\begin{RecordType}{\hyperB{IT::Substance}{Substance}}{}{}{Chemical substance.}
\KeyItem{\hyperB{Substance::name}{name}}{String (generic)}{\textlangle{\it obligatory }\textrangle}{}{Name of the substance.}
\KeyItem{\hyperB{Substance::molar-mass}{molar\_mass}}{String (generic)}{Double<NumberRange {0 , 1.79769e+308}>}{\textlangle{\it value at declaration }\textrangle}{}{Molar mass of the substance [kg/mol].}
\end{RecordType}

\begin{AbstractType}{\hyperB{IT::ReactionTerm}{ReactionTerm}}{}{}{Equation for reading information about simple chemical reactions.}
\Descendant{\Alink{IT::FirstOrderReaction}{FirstOrderReaction}}
\Descendant{\Alink{IT::RadioactiveDecay}{RadioactiveDecay}}
\Descendant{\Alink{IT::Sorption}{Sorption}}
\Descendant{\Alink{IT::SorptionMobile}{SorptionMobile}}
\Descendant{\Alink{IT::SorptionImmobile}{SorptionImmobile}}
\Descendant{\Alink{IT::DualPorosity}{DualPorosity}}
\Descendant{\Alink{IT::Semchem}{Semchem}}
\end{AbstractType}

\begin{RecordType}{\hyperB{IT::FirstOrderReaction}{FirstOrderReaction}}{\Alink{IT::ReactionTerm}{ReactionTerm}}{}{}{A model of first order chemical reactions (decompositions of a reactant into products).}
\KeyItem{\hyperB{FirstOrderReaction::TYPE}{TYPE}}{selection: ReactionTerm\_TYPE\_selection}{FirstOrderReaction}{}{Sub-record selection.}
\KeyItem{\hyperB{FirstOrderReaction::reactions}{reactions}}{String (generic)}{Array of HOW_TO_DETERMINE_TYPE}{\textlangle{\it obligatory }\textrangle}{}{An array of first order chemical reactions.}
\KeyItem{\hyperB{FirstOrderReaction::ode-solver}{ode\_solver}}{abstract type: \Alink{IT::LinearODESolver}{LinearODESolver}}{\textlangle{\it optional }\textrangle}{}{Numerical solver for the system of first order ordinary differential equations coming from the model.}
\end{RecordType}

\begin{SelectionType}{\hyperB{IT::ReactionTerm-TYPE-selection}{ReactionTerm\_TYPE\_selection}}{}
\KeyItem{FirstOrderReaction}{}
\KeyItem{RadioactiveDecay}{}
\KeyItem{Sorption}{}
\KeyItem{SorptionMobile}{}
\KeyItem{SorptionImmobile}{}
\KeyItem{DualPorosity}{}
\KeyItem{Semchem}{}
\end{SelectionType}

\begin{RecordType}{\hyperB{IT::Reaction}{Reaction}}{}{}{Describes a single first order chemical reaction.}
\KeyItem{\hyperB{Reaction::reactants}{reactants}}{String (generic)}{Array of HOW_TO_DETERMINE_TYPE}{\textlangle{\it obligatory }\textrangle}{}{An array of reactants. Do not use array, reactions with only one reactant (decays) are implemented at the moment!}
\KeyItem{\hyperB{Reaction::reaction-rate}{reaction\_rate}}{String (generic)}{Double<NumberRange {0 , 1.79769e+308}>}{\textlangle{\it obligatory }\textrangle}{}{The reaction rate coefficient of the first order reaction.}
\KeyItem{\hyperB{Reaction::products}{products}}{String (generic)}{Array of HOW_TO_DETERMINE_TYPE}{\textlangle{\it obligatory }\textrangle}{}{An array of products.}
\end{RecordType}

\begin{RecordType}{\hyperB{IT::FirstOrderReactionReactant}{FirstOrderReactionReactant}}{}{}{A record describing a reactant of a reaction.}
\KeyItem{\hyperB{FirstOrderReactionReactant::name}{name}}{String (generic)}{\textlangle{\it obligatory }\textrangle}{}{The name of the reactant.}
\end{RecordType}

\begin{RecordType}{\hyperB{IT::FirstOrderReactionProduct}{FirstOrderReactionProduct}}{}{}{A record describing a product of a reaction.}
\KeyItem{\hyperB{FirstOrderReactionProduct::name}{name}}{String (generic)}{\textlangle{\it obligatory }\textrangle}{}{The name of the product.}
\KeyItem{\hyperB{FirstOrderReactionProduct::branching-ratio}{branching\_ratio}}{String (generic)}{Double<NumberRange {0 , 1.79769e+308}>}{\textlangle{\it value at declaration }\textrangle}{}{The branching ratio of the product when there are more products.\\The value must be positive. Further, the branching ratios of all products are normalized in order to sum to one.\\The default value 1.0, should only be used in the case of single product.}
\end{RecordType}

\begin{AbstractType}{\hyperB{IT::LinearODESolver}{LinearODESolver}}{}{}{Solver of a linear system of ODEs.}
\Descendant{\Alink{IT::PadeApproximant}{PadeApproximant}}
\Descendant{\Alink{IT::LinearODEAnalytic}{LinearODEAnalytic}}
\end{AbstractType}

\begin{RecordType}{\hyperB{IT::PadeApproximant}{PadeApproximant}}{\Alink{IT::LinearODESolver}{LinearODESolver}}{}{}{Record with an information about pade approximant parameters.}
\KeyItem{\hyperB{PadeApproximant::TYPE}{TYPE}}{selection: LinearODESolver\_TYPE\_selection}{PadeApproximant}{}{Sub-record selection.}
\KeyItem{\hyperB{PadeApproximant::nominator-degree}{nominator\_degree}}{String (generic)}{Integer<NumberRange {1 , 2147483647}>}{\textlangle{\it value at declaration }\textrangle}{}{Polynomial degree of the nominator of Pade approximant.}
\KeyItem{\hyperB{PadeApproximant::denominator-degree}{denominator\_degree}}{String (generic)}{Integer<NumberRange {1 , 2147483647}>}{\textlangle{\it value at declaration }\textrangle}{}{Polynomial degree of the nominator of Pade approximant}
\end{RecordType}

\begin{SelectionType}{\hyperB{IT::LinearODESolver-TYPE-selection}{LinearODESolver\_TYPE\_selection}}{}
\KeyItem{PadeApproximant}{}
\KeyItem{LinearODEAnalytic}{}
\end{SelectionType}

\begin{RecordType}{\hyperB{IT::LinearODEAnalytic}{LinearODEAnalytic}}{\Alink{IT::LinearODESolver}{LinearODESolver}}{}{}{Evaluate analytic solution of the system of ODEs.}
\KeyItem{\hyperB{LinearODEAnalytic::TYPE}{TYPE}}{selection: LinearODESolver\_TYPE\_selection}{LinearODEAnalytic}{}{Sub-record selection.}
\end{RecordType}

\begin{RecordType}{\hyperB{IT::RadioactiveDecay}{RadioactiveDecay}}{\Alink{IT::ReactionTerm}{ReactionTerm}}{}{}{A model of a radioactive decay and possibly of a decay chain.}
\KeyItem{\hyperB{RadioactiveDecay::TYPE}{TYPE}}{selection: ReactionTerm\_TYPE\_selection}{RadioactiveDecay}{}{Sub-record selection.}
\KeyItem{\hyperB{RadioactiveDecay::decays}{decays}}{String (generic)}{Array of HOW_TO_DETERMINE_TYPE}{\textlangle{\it obligatory }\textrangle}{}{An array of radioactive decays.}
\KeyItem{\hyperB{RadioactiveDecay::ode-solver}{ode\_solver}}{abstract type: \Alink{IT::LinearODESolver}{LinearODESolver}}{\textlangle{\it optional }\textrangle}{}{Numerical solver for the system of first order ordinary differential equations coming from the model.}
\end{RecordType}

\begin{RecordType}{\hyperB{IT::Decay}{Decay}}{}{}{A model of a radioactive decay.}
\KeyItem{\hyperB{Decay::radionuclide}{radionuclide}}{String (generic)}{\textlangle{\it obligatory }\textrangle}{}{The name of the parent radionuclide.}
\KeyItem{\hyperB{Decay::half-life}{half\_life}}{String (generic)}{Double<NumberRange {0 , 1.79769e+308}>}{\textlangle{\it obligatory }\textrangle}{}{The half life of the parent radionuclide in seconds.}
\KeyItem{\hyperB{Decay::products}{products}}{String (generic)}{Array of HOW_TO_DETERMINE_TYPE}{\textlangle{\it obligatory }\textrangle}{}{An array of the decay products (daughters).}
\end{RecordType}

\begin{RecordType}{\hyperB{IT::RadioactiveDecayProduct}{RadioactiveDecayProduct}}{}{}{A record describing a product of a radioactive decay.}
\KeyItem{\hyperB{RadioactiveDecayProduct::name}{name}}{String (generic)}{\textlangle{\it obligatory }\textrangle}{}{The name of the product.}
\KeyItem{\hyperB{RadioactiveDecayProduct::energy}{energy}}{String (generic)}{Double<NumberRange {0 , 1.79769e+308}>}{\textlangle{\it value at declaration }\textrangle}{}{Not used at the moment! The released energy in MeV from the decay of the radionuclide into the product.}
\KeyItem{\hyperB{RadioactiveDecayProduct::branching-ratio}{branching\_ratio}}{String (generic)}{Double<NumberRange {0 , 1.79769e+308}>}{\textlangle{\it value at declaration }\textrangle}{}{The branching ratio of the product when there is more than one.Considering only one product, the default ratio 1.0 is used.Its value must be positive. Further, the branching ratios of all products are normalizedby their sum, so the sum then gives 1.0 (this also resolves possible rounding errors).}
\end{RecordType}

\begin{RecordType}{\hyperB{IT::Sorption}{Sorption}}{\Alink{IT::ReactionTerm}{ReactionTerm}}{}{}{Sorption model in the reaction term of transport.}
\KeyItem{\hyperB{Sorption::TYPE}{TYPE}}{selection: ReactionTerm\_TYPE\_selection}{Sorption}{}{Sub-record selection.}
\KeyItem{\hyperB{Sorption::substances}{substances}}{String (generic)}{Array of HOW_TO_DETERMINE_TYPE}{\textlangle{\it obligatory }\textrangle}{}{Names of the substances that take part in the sorption model.}
\KeyItem{\hyperB{Sorption::solvent-density}{solvent\_density}}{String (generic)}{Double<NumberRange {0 , 1.79769e+308}>}{\textlangle{\it value at declaration }\textrangle}{}{Density of the solvent.}
\KeyItem{\hyperB{Sorption::substeps}{substeps}}{String (generic)}{Integer<NumberRange {1 , 2147483647}>}{\textlangle{\it value at declaration }\textrangle}{}{Number of equidistant substeps, molar mass and isotherm intersections}
\KeyItem{\hyperB{Sorption::solubility}{solubility}}{String (generic)}{Array of HOW_TO_DETERMINE_TYPE}{\textlangle{\it optional }\textrangle}{}{Specifies solubility limits of all the sorbing species.}
\KeyItem{\hyperB{Sorption::table-limits}{table\_limits}}{String (generic)}{Array of HOW_TO_DETERMINE_TYPE}{\textlangle{\it optional }\textrangle}{}{Specifies highest aqueous concentration in interpolation table.}
\KeyItem{\hyperB{Sorption::input-fields}{input\_fields}}{String (generic)}{Array of HOW_TO_DETERMINE_TYPE}{\textlangle{\it obligatory }\textrangle}{}{Containes region specific data necessary to construct isotherms.}
\KeyItem{\hyperB{Sorption::reaction-liquid}{reaction\_liquid}}{abstract type: \Alink{IT::ReactionTerm}{ReactionTerm}}{\textlangle{\it optional }\textrangle}{}{Reaction model following the sorption in the liquid.}
\KeyItem{\hyperB{Sorption::reaction-solid}{reaction\_solid}}{abstract type: \Alink{IT::ReactionTerm}{ReactionTerm}}{\textlangle{\it optional }\textrangle}{}{Reaction model following the sorption in the solid.}
\KeyItem{\hyperB{Sorption::output-fields}{output\_fields}}{String (generic)}{Array of HOW_TO_DETERMINE_TYPE}{\textlangle{\it value at declaration }\textrangle}{}{List of fields to write to output stream.}
\end{RecordType}

\begin{RecordType}{\hyperB{IT::Sorption-Data}{Sorption\_Data}}{}{}{Record to set fields of the equation.\\The fields are set only on the domain specified by one of the keys: 'region', 'rid', 'r_set'\\and after the time given by the key 'time'. The field setting can be overridden by\\ any Sorption_Data record that comes later in the boundary data array.}
\KeyItem{\hyperB{Sorption-Data::r-set}{r\_set}}{String (generic)}{\textlangle{\it optional }\textrangle}{}{Name of region set where to set fields.}
\KeyItem{\hyperB{Sorption-Data::region}{region}}{String (generic)}{\textlangle{\it optional }\textrangle}{}{Label of the region where to set fields.}
\KeyItem{\hyperB{Sorption-Data::rid}{rid}}{String (generic)}{Integer<NumberRange {0 , 2147483647}>}{\textlangle{\it optional }\textrangle}{}{ID of the region where to set fields.}
\KeyItem{\hyperB{Sorption-Data::time}{time}}{String (generic)}{Double<NumberRange {0 , 1.79769e+308}>}{\textlangle{\it value at declaration }\textrangle}{}{Apply field setting in this record after this time.\\These times have to form an increasing sequence.}
\KeyItem{\hyperB{Sorption-Data::rock-density}{rock\_density}}{abstract type: \Alink{IT::Field:R3 -> Real}{Field:R3 -> Real}}{\textlangle{\it optional }\textrangle}{}{Rock matrix density. $[m^{-3}kg]$}
\KeyItem{\hyperB{Sorption-Data::sorption-type}{sorption\_type}}{abstract type: \Alink{IT::Field:R3 -> Enum[n]}{Field:R3 -> Enum[n]}}{\textlangle{\it optional }\textrangle}{}{Considered sorption is described by selected isotherm. If porosity on an element is equal or even higher than 1.0 (meaning no sorbing surface), then type 'none' will be selected automatically. $[-]$}
\KeyItem{\hyperB{Sorption-Data::isotherm-mult}{isotherm\_mult}}{abstract type: \Alink{IT::Field:R3 -> Real[n]}{Field:R3 -> Real[n]}}{\textlangle{\it optional }\textrangle}{}{Multiplication parameters (k, omega) in either Langmuir c_s = omega * (alpha*c_a)/(1- alpha*c_a) or in linear c_s = k * c_a isothermal description. $[kg^{-1}mol]$}
\KeyItem{\hyperB{Sorption-Data::isotherm-other}{isotherm\_other}}{abstract type: \Alink{IT::Field:R3 -> Real[n]}{Field:R3 -> Real[n]}}{\textlangle{\it optional }\textrangle}{}{Second parameters (alpha, ...) defining isotherm  c_s = omega * (alpha*c_a)/(1- alpha*c_a). $[-]$}
\KeyItem{\hyperB{Sorption-Data::init-conc-solid}{init\_conc\_solid}}{abstract type: \Alink{IT::Field:R3 -> Real[n]}{Field:R3 -> Real[n]}}{\textlangle{\it optional }\textrangle}{}{Initial solid concentration of substances. Vector, one value for every substance. $[kg^{-1}mol]$}
\end{RecordType}

\begin{AbstractType}{\hyperB{IT::Field:R3 -> Enum[n]}{Field:R3 -> Enum[n]}}{}{}{Abstract record for all time-space functions.}
\Descendant{\Alink{IT::FieldConstant}{FieldConstant}}
\Descendant{\Alink{IT::FieldFormula}{FieldFormula}}
\Descendant{\Alink{IT::FieldPython}{FieldPython}}
\Descendant{\Alink{IT::FieldInterpolatedP0}{FieldInterpolatedP0}}
\Descendant{\Alink{IT::FieldElementwise}{FieldElementwise}}
\end{AbstractType}

\begin{RecordType}{\hyperB{IT::FieldConstant}{FieldConstant}}{\Alink{IT::Field:R3 -> Enum[n]}{Field:R3 -> Enum[n]}}{}{}{R3 -> Enum[n] Field constant in space.}
\KeyItem{\hyperB{FieldConstant::TYPE}{TYPE}}{selection: Field:R3 -> Enum[n]\_TYPE\_selection}{FieldConstant}{}{Sub-record selection.}
\KeyItem{\hyperB{FieldConstant::value}{value}}{String (generic)}{Array of HOW_TO_DETERMINE_TYPE}{\textlangle{\it obligatory }\textrangle}{}{Value of the constant field.\\For vector values, you can use scalar value to enter constant vector.\\For square NxN-matrix values, you can use:\\* vector of size N to enter diagonal matrix\\* vector of size (N+1)*N/2 to enter symmetric matrix (upper triangle, row by row)\\* scalar to enter multiple of the unit matrix.}
\end{RecordType}

\begin{SelectionType}{\hyperB{IT::Field:R3 -> Enum[n]-TYPE-selection}{Field:R3 -> Enum[n]\_TYPE\_selection}}{}
\KeyItem{FieldConstant}{}
\KeyItem{FieldFormula}{}
\KeyItem{FieldPython}{}
\KeyItem{FieldInterpolatedP0}{}
\KeyItem{FieldElementwise}{}
\end{SelectionType}

\begin{SelectionType}{\hyperB{IT::SorptionType}{SorptionType}}{}
\KeyItem{none}{}
\KeyItem{linear}{}
\KeyItem{langmuir}{}
\KeyItem{freundlich}{}
\end{SelectionType}

\begin{RecordType}{\hyperB{IT::FieldFormula}{FieldFormula}}{\Alink{IT::Field:R3 -> Enum[n]}{Field:R3 -> Enum[n]}}{}{}{R3 -> Enum[n] Field given by runtime interpreted formula.}
\KeyItem{\hyperB{FieldFormula::TYPE}{TYPE}}{selection: Field:R3 -> Enum[n]\_TYPE\_selection}{FieldFormula}{}{Sub-record selection.}
\KeyItem{\hyperB{FieldFormula::value}{value}}{String (generic)}{Array of HOW_TO_DETERMINE_TYPE}{\textlangle{\it obligatory }\textrangle}{}{String, array of strings, or matrix of strings with formulas for individual entries of scalar, vector, or tensor value respectively.\\For vector values, you can use just one string to enter homogeneous vector.\\For square NxN-matrix values, you can use:\\* array of strings of size N to enter diagonal matrix\\* array of strings of size (N+1)*N/2 to enter symmetric matrix (upper triangle, row by row)\\* just one string to enter (spatially variable) multiple of the unit matrix.\\Formula can contain variables x,y,z,t and usual operators and functions.}
\end{RecordType}

\begin{RecordType}{\hyperB{IT::FieldPython}{FieldPython}}{\Alink{IT::Field:R3 -> Enum[n]}{Field:R3 -> Enum[n]}}{}{}{R3 -> Enum[n] Field given by a Python script.}
\KeyItem{\hyperB{FieldPython::TYPE}{TYPE}}{selection: Field:R3 -> Enum[n]\_TYPE\_selection}{FieldPython}{}{Sub-record selection.}
\KeyItem{\hyperB{FieldPython::script-string}{script\_string}}{String (generic)}{\textlangle{\it value at read time }\textrangle}{}{Python script given as in place string}
\KeyItem{\hyperB{FieldPython::script-file}{script\_file}}{String (generic)}{input file name}{\textlangle{\it value at read time }\textrangle}{}{Python script given as external file}
\KeyItem{\hyperB{FieldPython::function}{function}}{String (generic)}{\textlangle{\it obligatory }\textrangle}{}{Function in the given script that returns tuple containing components of the return type.\\For NxM tensor values: tensor(row,col) = tuple( M*row + col ).}
\end{RecordType}

\begin{RecordType}{\hyperB{IT::FieldInterpolatedP0}{FieldInterpolatedP0}}{\Alink{IT::Field:R3 -> Enum[n]}{Field:R3 -> Enum[n]}}{}{}{R3 -> Enum[n] Field constant in space.}
\KeyItem{\hyperB{FieldInterpolatedP0::TYPE}{TYPE}}{selection: Field:R3 -> Enum[n]\_TYPE\_selection}{FieldInterpolatedP0}{}{Sub-record selection.}
\KeyItem{\hyperB{FieldInterpolatedP0::gmsh-file}{gmsh\_file}}{String (generic)}{input file name}{\textlangle{\it obligatory }\textrangle}{}{Input file with ASCII GMSH file format.}
\KeyItem{\hyperB{FieldInterpolatedP0::field-name}{field\_name}}{String (generic)}{\textlangle{\it obligatory }\textrangle}{}{The values of the Field are read from the $ElementData section with field name given by this key.}
\end{RecordType}

\begin{RecordType}{\hyperB{IT::FieldElementwise}{FieldElementwise}}{\Alink{IT::Field:R3 -> Enum[n]}{Field:R3 -> Enum[n]}}{}{}{R3 -> Enum[n] Field constant in space.}
\KeyItem{\hyperB{FieldElementwise::TYPE}{TYPE}}{selection: Field:R3 -> Enum[n]\_TYPE\_selection}{FieldElementwise}{}{Sub-record selection.}
\KeyItem{\hyperB{FieldElementwise::gmsh-file}{gmsh\_file}}{String (generic)}{input file name}{\textlangle{\it obligatory }\textrangle}{}{Input file with ASCII GMSH file format.}
\KeyItem{\hyperB{FieldElementwise::field-name}{field\_name}}{String (generic)}{\textlangle{\it obligatory }\textrangle}{}{The values of the Field are read from the $ElementData section with field name given by this key.}
\end{RecordType}

\begin{AbstractType}{\hyperB{IT::Field:R3 -> Real[n]}{Field:R3 -> Real[n]}}{}{}{Abstract record for all time-space functions.}
\Descendant{\Alink{IT::FieldConstant}{FieldConstant}}
\Descendant{\Alink{IT::FieldPython}{FieldPython}}
\Descendant{\Alink{IT::FieldFormula}{FieldFormula}}
\Descendant{\Alink{IT::FieldElementwise}{FieldElementwise}}
\Descendant{\Alink{IT::FieldInterpolatedP0}{FieldInterpolatedP0}}
\end{AbstractType}

\begin{RecordType}{\hyperB{IT::FieldConstant}{FieldConstant}}{\Alink{IT::Field:R3 -> Real[n]}{Field:R3 -> Real[n]}}{}{}{R3 -> Real[n] Field constant in space.}
\KeyItem{\hyperB{FieldConstant::TYPE}{TYPE}}{selection: Field:R3 -> Real[n]\_TYPE\_selection}{FieldConstant}{}{Sub-record selection.}
\KeyItem{\hyperB{FieldConstant::value}{value}}{String (generic)}{Array of HOW_TO_DETERMINE_TYPE}{\textlangle{\it obligatory }\textrangle}{}{Value of the constant field.\\For vector values, you can use scalar value to enter constant vector.\\For square NxN-matrix values, you can use:\\* vector of size N to enter diagonal matrix\\* vector of size (N+1)*N/2 to enter symmetric matrix (upper triangle, row by row)\\* scalar to enter multiple of the unit matrix.}
\end{RecordType}

\begin{SelectionType}{\hyperB{IT::Field:R3 -> Real[n]-TYPE-selection}{Field:R3 -> Real[n]\_TYPE\_selection}}{}
\KeyItem{FieldConstant}{}
\KeyItem{FieldPython}{}
\KeyItem{FieldFormula}{}
\KeyItem{FieldElementwise}{}
\KeyItem{FieldInterpolatedP0}{}
\end{SelectionType}

\begin{RecordType}{\hyperB{IT::FieldPython}{FieldPython}}{\Alink{IT::Field:R3 -> Real[n]}{Field:R3 -> Real[n]}}{}{}{R3 -> Real[n] Field given by a Python script.}
\KeyItem{\hyperB{FieldPython::TYPE}{TYPE}}{selection: Field:R3 -> Real[n]\_TYPE\_selection}{FieldPython}{}{Sub-record selection.}
\KeyItem{\hyperB{FieldPython::script-string}{script\_string}}{String (generic)}{\textlangle{\it value at read time }\textrangle}{}{Python script given as in place string}
\KeyItem{\hyperB{FieldPython::script-file}{script\_file}}{String (generic)}{input file name}{\textlangle{\it value at read time }\textrangle}{}{Python script given as external file}
\KeyItem{\hyperB{FieldPython::function}{function}}{String (generic)}{\textlangle{\it obligatory }\textrangle}{}{Function in the given script that returns tuple containing components of the return type.\\For NxM tensor values: tensor(row,col) = tuple( M*row + col ).}
\end{RecordType}

\begin{RecordType}{\hyperB{IT::FieldFormula}{FieldFormula}}{\Alink{IT::Field:R3 -> Real[n]}{Field:R3 -> Real[n]}}{}{}{R3 -> Real[n] Field given by runtime interpreted formula.}
\KeyItem{\hyperB{FieldFormula::TYPE}{TYPE}}{selection: Field:R3 -> Real[n]\_TYPE\_selection}{FieldFormula}{}{Sub-record selection.}
\KeyItem{\hyperB{FieldFormula::value}{value}}{String (generic)}{Array of HOW_TO_DETERMINE_TYPE}{\textlangle{\it obligatory }\textrangle}{}{String, array of strings, or matrix of strings with formulas for individual entries of scalar, vector, or tensor value respectively.\\For vector values, you can use just one string to enter homogeneous vector.\\For square NxN-matrix values, you can use:\\* array of strings of size N to enter diagonal matrix\\* array of strings of size (N+1)*N/2 to enter symmetric matrix (upper triangle, row by row)\\* just one string to enter (spatially variable) multiple of the unit matrix.\\Formula can contain variables x,y,z,t and usual operators and functions.}
\end{RecordType}

\begin{RecordType}{\hyperB{IT::FieldElementwise}{FieldElementwise}}{\Alink{IT::Field:R3 -> Real[n]}{Field:R3 -> Real[n]}}{}{}{R3 -> Real[n] Field constant in space.}
\KeyItem{\hyperB{FieldElementwise::TYPE}{TYPE}}{selection: Field:R3 -> Real[n]\_TYPE\_selection}{FieldElementwise}{}{Sub-record selection.}
\KeyItem{\hyperB{FieldElementwise::gmsh-file}{gmsh\_file}}{String (generic)}{input file name}{\textlangle{\it obligatory }\textrangle}{}{Input file with ASCII GMSH file format.}
\KeyItem{\hyperB{FieldElementwise::field-name}{field\_name}}{String (generic)}{\textlangle{\it obligatory }\textrangle}{}{The values of the Field are read from the $ElementData section with field name given by this key.}
\end{RecordType}

\begin{RecordType}{\hyperB{IT::FieldInterpolatedP0}{FieldInterpolatedP0}}{\Alink{IT::Field:R3 -> Real[n]}{Field:R3 -> Real[n]}}{}{}{R3 -> Real[n] Field constant in space.}
\KeyItem{\hyperB{FieldInterpolatedP0::TYPE}{TYPE}}{selection: Field:R3 -> Real[n]\_TYPE\_selection}{FieldInterpolatedP0}{}{Sub-record selection.}
\KeyItem{\hyperB{FieldInterpolatedP0::gmsh-file}{gmsh\_file}}{String (generic)}{input file name}{\textlangle{\it obligatory }\textrangle}{}{Input file with ASCII GMSH file format.}
\KeyItem{\hyperB{FieldInterpolatedP0::field-name}{field\_name}}{String (generic)}{\textlangle{\it obligatory }\textrangle}{}{The values of the Field are read from the $ElementData section with field name given by this key.}
\end{RecordType}

\begin{SelectionType}{\hyperB{IT::Sorption-Output}{Sorption\_Output}}{}
\KeyItem{rock_density}{}
\KeyItem{sorption_type}{}
\KeyItem{isotherm_mult}{}
\KeyItem{isotherm_other}{}
\KeyItem{init_conc_solid}{}
\KeyItem{conc_solid}{}
\end{SelectionType}

\begin{RecordType}{\hyperB{IT::SorptionMobile}{SorptionMobile}}{\Alink{IT::ReactionTerm}{ReactionTerm}}{}{}{Sorption model in the mobile zone, following the dual porosity model.}
\KeyItem{\hyperB{SorptionMobile::TYPE}{TYPE}}{selection: ReactionTerm\_TYPE\_selection}{SorptionMobile}{}{Sub-record selection.}
\KeyItem{\hyperB{SorptionMobile::substances}{substances}}{String (generic)}{Array of HOW_TO_DETERMINE_TYPE}{\textlangle{\it obligatory }\textrangle}{}{Names of the substances that take part in the sorption model.}
\KeyItem{\hyperB{SorptionMobile::solvent-density}{solvent\_density}}{String (generic)}{Double<NumberRange {0 , 1.79769e+308}>}{\textlangle{\it value at declaration }\textrangle}{}{Density of the solvent.}
\KeyItem{\hyperB{SorptionMobile::substeps}{substeps}}{String (generic)}{Integer<NumberRange {1 , 2147483647}>}{\textlangle{\it value at declaration }\textrangle}{}{Number of equidistant substeps, molar mass and isotherm intersections}
\KeyItem{\hyperB{SorptionMobile::solubility}{solubility}}{String (generic)}{Array of HOW_TO_DETERMINE_TYPE}{\textlangle{\it optional }\textrangle}{}{Specifies solubility limits of all the sorbing species.}
\KeyItem{\hyperB{SorptionMobile::table-limits}{table\_limits}}{String (generic)}{Array of HOW_TO_DETERMINE_TYPE}{\textlangle{\it optional }\textrangle}{}{Specifies highest aqueous concentration in interpolation table.}
\KeyItem{\hyperB{SorptionMobile::input-fields}{input\_fields}}{String (generic)}{Array of HOW_TO_DETERMINE_TYPE}{\textlangle{\it obligatory }\textrangle}{}{Containes region specific data necessary to construct isotherms.}
\KeyItem{\hyperB{SorptionMobile::reaction-liquid}{reaction\_liquid}}{abstract type: \Alink{IT::ReactionTerm}{ReactionTerm}}{\textlangle{\it optional }\textrangle}{}{Reaction model following the sorption in the liquid.}
\KeyItem{\hyperB{SorptionMobile::reaction-solid}{reaction\_solid}}{abstract type: \Alink{IT::ReactionTerm}{ReactionTerm}}{\textlangle{\it optional }\textrangle}{}{Reaction model following the sorption in the solid.}
\KeyItem{\hyperB{SorptionMobile::output-fields}{output\_fields}}{String (generic)}{Array of HOW_TO_DETERMINE_TYPE}{\textlangle{\it value at declaration }\textrangle}{}{List of fields to write to output stream.}
\end{RecordType}

\begin{SelectionType}{\hyperB{IT::SorptionMobile-Output}{SorptionMobile\_Output}}{}
\KeyItem{rock_density}{}
\KeyItem{sorption_type}{}
\KeyItem{isotherm_mult}{}
\KeyItem{isotherm_other}{}
\KeyItem{init_conc_solid}{}
\KeyItem{conc_solid}{}
\end{SelectionType}

\begin{RecordType}{\hyperB{IT::SorptionImmobile}{SorptionImmobile}}{\Alink{IT::ReactionTerm}{ReactionTerm}}{}{}{Sorption model in the immobile zone, following the dual porosity model.}
\KeyItem{\hyperB{SorptionImmobile::TYPE}{TYPE}}{selection: ReactionTerm\_TYPE\_selection}{SorptionImmobile}{}{Sub-record selection.}
\KeyItem{\hyperB{SorptionImmobile::substances}{substances}}{String (generic)}{Array of HOW_TO_DETERMINE_TYPE}{\textlangle{\it obligatory }\textrangle}{}{Names of the substances that take part in the sorption model.}
\KeyItem{\hyperB{SorptionImmobile::solvent-density}{solvent\_density}}{String (generic)}{Double<NumberRange {0 , 1.79769e+308}>}{\textlangle{\it value at declaration }\textrangle}{}{Density of the solvent.}
\KeyItem{\hyperB{SorptionImmobile::substeps}{substeps}}{String (generic)}{Integer<NumberRange {1 , 2147483647}>}{\textlangle{\it value at declaration }\textrangle}{}{Number of equidistant substeps, molar mass and isotherm intersections}
\KeyItem{\hyperB{SorptionImmobile::solubility}{solubility}}{String (generic)}{Array of HOW_TO_DETERMINE_TYPE}{\textlangle{\it optional }\textrangle}{}{Specifies solubility limits of all the sorbing species.}
\KeyItem{\hyperB{SorptionImmobile::table-limits}{table\_limits}}{String (generic)}{Array of HOW_TO_DETERMINE_TYPE}{\textlangle{\it optional }\textrangle}{}{Specifies highest aqueous concentration in interpolation table.}
\KeyItem{\hyperB{SorptionImmobile::input-fields}{input\_fields}}{String (generic)}{Array of HOW_TO_DETERMINE_TYPE}{\textlangle{\it obligatory }\textrangle}{}{Containes region specific data necessary to construct isotherms.}
\KeyItem{\hyperB{SorptionImmobile::reaction-liquid}{reaction\_liquid}}{abstract type: \Alink{IT::ReactionTerm}{ReactionTerm}}{\textlangle{\it optional }\textrangle}{}{Reaction model following the sorption in the liquid.}
\KeyItem{\hyperB{SorptionImmobile::reaction-solid}{reaction\_solid}}{abstract type: \Alink{IT::ReactionTerm}{ReactionTerm}}{\textlangle{\it optional }\textrangle}{}{Reaction model following the sorption in the solid.}
\KeyItem{\hyperB{SorptionImmobile::output-fields}{output\_fields}}{String (generic)}{Array of HOW_TO_DETERMINE_TYPE}{\textlangle{\it value at declaration }\textrangle}{}{List of fields to write to output stream.}
\end{RecordType}

\begin{SelectionType}{\hyperB{IT::SorptionImmobile-Output}{SorptionImmobile\_Output}}{}
\KeyItem{rock_density}{}
\KeyItem{sorption_type}{}
\KeyItem{isotherm_mult}{}
\KeyItem{isotherm_other}{}
\KeyItem{init_conc_solid}{}
\KeyItem{conc_immobile_solid}{}
\end{SelectionType}

\begin{RecordType}{\hyperB{IT::DualPorosity}{DualPorosity}}{\Alink{IT::ReactionTerm}{ReactionTerm}}{}{}{Dual porosity model in transport problems.\\Provides computing the concentration of substances in mobile and immobile zone.}
\KeyItem{\hyperB{DualPorosity::TYPE}{TYPE}}{selection: ReactionTerm\_TYPE\_selection}{DualPorosity}{}{Sub-record selection.}
\KeyItem{\hyperB{DualPorosity::input-fields}{input\_fields}}{String (generic)}{Array of HOW_TO_DETERMINE_TYPE}{\textlangle{\it obligatory }\textrangle}{}{Containes region specific data necessary to construct dual porosity model.}
\KeyItem{\hyperB{DualPorosity::scheme-tolerance}{scheme\_tolerance}}{String (generic)}{Double<NumberRange {0 , 1.79769e+308}>}{\textlangle{\it value at declaration }\textrangle}{}{Tolerance according to which the explicit Euler scheme is used or not.Set 0.0 to use analytic formula only (can be slower).}
\KeyItem{\hyperB{DualPorosity::reaction-mobile}{reaction\_mobile}}{abstract type: \Alink{IT::ReactionTerm}{ReactionTerm}}{\textlangle{\it optional }\textrangle}{}{Reaction model in mobile zone.}
\KeyItem{\hyperB{DualPorosity::reaction-immobile}{reaction\_immobile}}{abstract type: \Alink{IT::ReactionTerm}{ReactionTerm}}{\textlangle{\it optional }\textrangle}{}{Reaction model in immobile zone.}
\KeyItem{\hyperB{DualPorosity::output-fields}{output\_fields}}{String (generic)}{Array of HOW_TO_DETERMINE_TYPE}{\textlangle{\it value at declaration }\textrangle}{}{List of fields to write to output stream.}
\end{RecordType}

\begin{RecordType}{\hyperB{IT::DualPorosity-Data}{DualPorosity\_Data}}{}{}{Record to set fields of the equation.\\The fields are set only on the domain specified by one of the keys: 'region', 'rid', 'r_set'\\and after the time given by the key 'time'. The field setting can be overridden by\\ any DualPorosity_Data record that comes later in the boundary data array.}
\KeyItem{\hyperB{DualPorosity-Data::r-set}{r\_set}}{String (generic)}{\textlangle{\it optional }\textrangle}{}{Name of region set where to set fields.}
\KeyItem{\hyperB{DualPorosity-Data::region}{region}}{String (generic)}{\textlangle{\it optional }\textrangle}{}{Label of the region where to set fields.}
\KeyItem{\hyperB{DualPorosity-Data::rid}{rid}}{String (generic)}{Integer<NumberRange {0 , 2147483647}>}{\textlangle{\it optional }\textrangle}{}{ID of the region where to set fields.}
\KeyItem{\hyperB{DualPorosity-Data::time}{time}}{String (generic)}{Double<NumberRange {0 , 1.79769e+308}>}{\textlangle{\it value at declaration }\textrangle}{}{Apply field setting in this record after this time.\\These times have to form an increasing sequence.}
\KeyItem{\hyperB{DualPorosity-Data::diffusion-rate-immobile}{diffusion\_rate\_immobile}}{abstract type: \Alink{IT::Field:R3 -> Real[n]}{Field:R3 -> Real[n]}}{\textlangle{\it optional }\textrangle}{}{Diffusion coefficient of non-equilibrium linear exchange between mobile and immobile zone. $[s^{-1}]$}
\KeyItem{\hyperB{DualPorosity-Data::porosity-immobile}{porosity\_immobile}}{abstract type: \Alink{IT::Field:R3 -> Real}{Field:R3 -> Real}}{\textlangle{\it optional }\textrangle}{}{Porosity of the immobile zone. $[-]$}
\KeyItem{\hyperB{DualPorosity-Data::init-conc-immobile}{init\_conc\_immobile}}{abstract type: \Alink{IT::Field:R3 -> Real[n]}{Field:R3 -> Real[n]}}{\textlangle{\it optional }\textrangle}{}{Initial concentration of substances in the immobile zone. $[m^{-3}kg]$}
\end{RecordType}

\begin{SelectionType}{\hyperB{IT::DualPorosity-Output}{DualPorosity\_Output}}{}
\KeyItem{diffusion_rate_immobile}{}
\KeyItem{porosity_immobile}{}
\KeyItem{init_conc_immobile}{}
\KeyItem{conc_immobile}{}
\end{SelectionType}

\begin{RecordType}{\hyperB{IT::Semchem}{Semchem}}{\Alink{IT::ReactionTerm}{ReactionTerm}}{}{}{Declares infos valid for all reactions. NOT SUPPORTED!!!.}
\KeyItem{\hyperB{Semchem::TYPE}{TYPE}}{selection: ReactionTerm\_TYPE\_selection}{Semchem}{}{Sub-record selection.}
\KeyItem{\hyperB{Semchem::precision}{precision}}{String (generic)}{Integer<NumberRange {-2147483648 , 2147483647}>}{\textlangle{\it obligatory }\textrangle}{}{How accurate should the simulation be, decimal places(?).}
\KeyItem{\hyperB{Semchem::temperature}{temperature}}{String (generic)}{Double<NumberRange {-1.79769e+308 , 1.79769e+308}>}{\textlangle{\it obligatory }\textrangle}{}{Isothermal reaction, thermodynamic temperature.}
\KeyItem{\hyperB{Semchem::temp-gf}{temp\_gf}}{String (generic)}{Double<NumberRange {-1.79769e+308 , 1.79769e+308}>}{\textlangle{\it obligatory }\textrangle}{}{Thermodynamic parameter.}
\KeyItem{\hyperB{Semchem::param-afi}{param\_afi}}{String (generic)}{Double<NumberRange {-1.79769e+308 , 1.79769e+308}>}{\textlangle{\it obligatory }\textrangle}{}{Thermodynamic parameter.}
\KeyItem{\hyperB{Semchem::param-b}{param\_b}}{String (generic)}{Double<NumberRange {-1.79769e+308 , 1.79769e+308}>}{\textlangle{\it obligatory }\textrangle}{}{Thermodynamic parameter.}
\KeyItem{\hyperB{Semchem::epsilon}{epsilon}}{String (generic)}{Double<NumberRange {-1.79769e+308 , 1.79769e+308}>}{\textlangle{\it obligatory }\textrangle}{}{Thermodynamic parameter.}
\KeyItem{\hyperB{Semchem::time-steps}{time\_steps}}{String (generic)}{Integer<NumberRange {-2147483648 , 2147483647}>}{\textlangle{\it obligatory }\textrangle}{}{Simulation parameter.}
\KeyItem{\hyperB{Semchem::slow-kinetic-steps}{slow\_kinetic\_steps}}{String (generic)}{Integer<NumberRange {-2147483648 , 2147483647}>}{\textlangle{\it obligatory }\textrangle}{}{Simulation parameter.}
\end{RecordType}

\begin{RecordType}{\hyperB{IT::TransportOperatorSplitting-Data}{TransportOperatorSplitting\_Data}}{}{}{Record to set fields of the equation.\\The fields are set only on the domain specified by one of the keys: 'region', 'rid', 'r_set'\\and after the time given by the key 'time'. The field setting can be overridden by\\ any TransportOperatorSplitting_Data record that comes later in the boundary data array.}
\KeyItem{\hyperB{TransportOperatorSplitting-Data::r-set}{r\_set}}{String (generic)}{\textlangle{\it optional }\textrangle}{}{Name of region set where to set fields.}
\KeyItem{\hyperB{TransportOperatorSplitting-Data::region}{region}}{String (generic)}{\textlangle{\it optional }\textrangle}{}{Label of the region where to set fields.}
\KeyItem{\hyperB{TransportOperatorSplitting-Data::rid}{rid}}{String (generic)}{Integer<NumberRange {0 , 2147483647}>}{\textlangle{\it optional }\textrangle}{}{ID of the region where to set fields.}
\KeyItem{\hyperB{TransportOperatorSplitting-Data::time}{time}}{String (generic)}{Double<NumberRange {0 , 1.79769e+308}>}{\textlangle{\it value at declaration }\textrangle}{}{Apply field setting in this record after this time.\\These times have to form an increasing sequence.}
\KeyItem{\hyperB{TransportOperatorSplitting-Data::porosity}{porosity}}{abstract type: \Alink{IT::Field:R3 -> Real}{Field:R3 -> Real}}{\textlangle{\it optional }\textrangle}{}{Mobile porosity $[-]$}
\KeyItem{\hyperB{TransportOperatorSplitting-Data::sources-density}{sources\_density}}{abstract type: \Alink{IT::Field:R3 -> Real[n]}{Field:R3 -> Real[n]}}{\textlangle{\it optional }\textrangle}{}{Density of concentration sources. $[m^{-3}kgs^{-1}]$}
\KeyItem{\hyperB{TransportOperatorSplitting-Data::sources-sigma}{sources\_sigma}}{abstract type: \Alink{IT::Field:R3 -> Real[n]}{Field:R3 -> Real[n]}}{\textlangle{\it optional }\textrangle}{}{Concentration flux. $[s^{-1}]$}
\KeyItem{\hyperB{TransportOperatorSplitting-Data::sources-conc}{sources\_conc}}{abstract type: \Alink{IT::Field:R3 -> Real[n]}{Field:R3 -> Real[n]}}{\textlangle{\it optional }\textrangle}{}{Concentration sources threshold. $[m^{-3}kg]$}
\KeyItem{\hyperB{TransportOperatorSplitting-Data::bc-conc}{bc\_conc}}{abstract type: \Alink{IT::Field:R3 -> Real[n]}{Field:R3 -> Real[n]}}{\textlangle{\it optional }\textrangle}{}{Boundary conditions for concentrations. $[m^{-3}kg]$}
\KeyItem{\hyperB{TransportOperatorSplitting-Data::init-conc}{init\_conc}}{abstract type: \Alink{IT::Field:R3 -> Real[n]}{Field:R3 -> Real[n]}}{\textlangle{\it optional }\textrangle}{}{Initial concentrations. $[m^{-3}kg]$}
\KeyItem{\hyperB{TransportOperatorSplitting-Data::transport-old-bcd-file}{transport\_old\_bcd\_file}}{String (generic)}{input file name}{\textlangle{\it optional }\textrangle}{}{File with mesh dependent boundary conditions (obsolete).}
\end{RecordType}

\begin{SelectionType}{\hyperB{IT::ConvectionTransport-Output}{ConvectionTransport\_Output}}{}
\KeyItem{porosity}{}
\KeyItem{sources_density}{}
\KeyItem{sources_sigma}{}
\KeyItem{sources_conc}{}
\KeyItem{init_conc}{}
\KeyItem{conc}{}
\KeyItem{region_id}{}
\end{SelectionType}

\begin{RecordType}{\hyperB{IT::SoluteTransport-DG}{SoluteTransport\_DG}}{\Alink{IT::Transport}{Transport}}{}{}{DG solver for solute transport.}
\KeyItem{\hyperB{SoluteTransport-DG::TYPE}{TYPE}}{selection: Transport\_TYPE\_selection}{SoluteTransport_DG}{}{Sub-record selection.}
\KeyItem{\hyperB{SoluteTransport-DG::time}{time}}{record: \Alink{IT::TimeGovernor}{TimeGovernor}}{\textlangle{\it obligatory }\textrangle}{}{Time governor setting for the secondary equation.}
\KeyItem{\hyperB{SoluteTransport-DG::balance}{balance}}{record: \Alink{IT::Balance}{Balance}}{\textlangle{\it obligatory }\textrangle}{}{Settings for computing balance.}
\KeyItem{\hyperB{SoluteTransport-DG::output-stream}{output\_stream}}{record: \Alink{IT::OutputStream}{OutputStream}}{\textlangle{\it obligatory }\textrangle}{}{Parameters of output stream.}
\KeyItem{\hyperB{SoluteTransport-DG::substances}{substances}}{String (generic)}{Array of HOW_TO_DETERMINE_TYPE}{\textlangle{\it obligatory }\textrangle}{}{Names of transported substances.}
\KeyItem{\hyperB{SoluteTransport-DG::solver}{solver}}{record: \Alink{IT::Petsc}{Petsc}}{\textlangle{\it obligatory }\textrangle}{}{Linear solver for MH problem.}
\KeyItem{\hyperB{SoluteTransport-DG::input-fields}{input\_fields}}{String (generic)}{Array of HOW_TO_DETERMINE_TYPE}{\textlangle{\it obligatory }\textrangle}{}{}
\KeyItem{\hyperB{SoluteTransport-DG::dg-variant}{dg\_variant}}{selection: DG\_variant}{non-symmetric}{}{Variant of interior penalty discontinuous Galerkin method.}
\KeyItem{\hyperB{SoluteTransport-DG::dg-order}{dg\_order}}{String (generic)}{Integer<NumberRange {0 , 3}>}{\textlangle{\it value at declaration }\textrangle}{}{Polynomial order for finite element in DG method (order 0 is suitable if there is no diffusion/dispersion).}
\KeyItem{\hyperB{SoluteTransport-DG::output-fields}{output\_fields}}{String (generic)}{Array of HOW_TO_DETERMINE_TYPE}{\textlangle{\it value at declaration }\textrangle}{}{List of fields to write to output file.}
\end{RecordType}

\begin{RecordType}{\hyperB{IT::SoluteTransport-DG-Data}{SoluteTransport\_DG\_Data}}{}{}{Record to set fields of the equation.\\The fields are set only on the domain specified by one of the keys: 'region', 'rid', 'r_set'\\and after the time given by the key 'time'. The field setting can be overridden by\\ any SoluteTransport_DG_Data record that comes later in the boundary data array.}
\KeyItem{\hyperB{SoluteTransport-DG-Data::r-set}{r\_set}}{String (generic)}{\textlangle{\it optional }\textrangle}{}{Name of region set where to set fields.}
\KeyItem{\hyperB{SoluteTransport-DG-Data::region}{region}}{String (generic)}{\textlangle{\it optional }\textrangle}{}{Label of the region where to set fields.}
\KeyItem{\hyperB{SoluteTransport-DG-Data::rid}{rid}}{String (generic)}{Integer<NumberRange {0 , 2147483647}>}{\textlangle{\it optional }\textrangle}{}{ID of the region where to set fields.}
\KeyItem{\hyperB{SoluteTransport-DG-Data::time}{time}}{String (generic)}{Double<NumberRange {0 , 1.79769e+308}>}{\textlangle{\it value at declaration }\textrangle}{}{Apply field setting in this record after this time.\\These times have to form an increasing sequence.}
\KeyItem{\hyperB{SoluteTransport-DG-Data::porosity}{porosity}}{abstract type: \Alink{IT::Field:R3 -> Real}{Field:R3 -> Real}}{\textlangle{\it optional }\textrangle}{}{Mobile porosity $[-]$}
\KeyItem{\hyperB{SoluteTransport-DG-Data::sources-density}{sources\_density}}{abstract type: \Alink{IT::Field:R3 -> Real[n]}{Field:R3 -> Real[n]}}{\textlangle{\it optional }\textrangle}{}{Density of concentration sources. $[m^{-3}kgs^{-1}]$}
\KeyItem{\hyperB{SoluteTransport-DG-Data::sources-sigma}{sources\_sigma}}{abstract type: \Alink{IT::Field:R3 -> Real[n]}{Field:R3 -> Real[n]}}{\textlangle{\it optional }\textrangle}{}{Concentration flux. $[s^{-1}]$}
\KeyItem{\hyperB{SoluteTransport-DG-Data::sources-conc}{sources\_conc}}{abstract type: \Alink{IT::Field:R3 -> Real[n]}{Field:R3 -> Real[n]}}{\textlangle{\it optional }\textrangle}{}{Concentration sources threshold. $[m^{-3}kg]$}
\KeyItem{\hyperB{SoluteTransport-DG-Data::bc-conc}{bc\_conc}}{abstract type: \Alink{IT::Field:R3 -> Real[n]}{Field:R3 -> Real[n]}}{\textlangle{\it optional }\textrangle}{}{Dirichlet boundary condition (for each substance). $[m^{-3}kg]$}
\KeyItem{\hyperB{SoluteTransport-DG-Data::init-conc}{init\_conc}}{abstract type: \Alink{IT::Field:R3 -> Real[n]}{Field:R3 -> Real[n]}}{\textlangle{\it optional }\textrangle}{}{Initial concentrations. $[m^{-3}kg]$}
\KeyItem{\hyperB{SoluteTransport-DG-Data::disp-l}{disp\_l}}{abstract type: \Alink{IT::Field:R3 -> Real[n]}{Field:R3 -> Real[n]}}{\textlangle{\it optional }\textrangle}{}{Longitudal dispersivity (for each substance). $[m]$}
\KeyItem{\hyperB{SoluteTransport-DG-Data::disp-t}{disp\_t}}{abstract type: \Alink{IT::Field:R3 -> Real[n]}{Field:R3 -> Real[n]}}{\textlangle{\it optional }\textrangle}{}{Transversal dispersivity (for each substance). $[m]$}
\KeyItem{\hyperB{SoluteTransport-DG-Data::diff-m}{diff\_m}}{abstract type: \Alink{IT::Field:R3 -> Real[n]}{Field:R3 -> Real[n]}}{\textlangle{\it optional }\textrangle}{}{Molecular diffusivity (for each substance). $[m^{2}s^{-1}]$}
\KeyItem{\hyperB{SoluteTransport-DG-Data::fracture-sigma}{fracture\_sigma}}{abstract type: \Alink{IT::Field:R3 -> Real[n]}{Field:R3 -> Real[n]}}{\textlangle{\it optional }\textrangle}{}{Coefficient of diffusive transfer through fractures (for each substance). $[-]$}
\KeyItem{\hyperB{SoluteTransport-DG-Data::dg-penalty}{dg\_penalty}}{abstract type: \Alink{IT::Field:R3 -> Real[n]}{Field:R3 -> Real[n]}}{\textlangle{\it optional }\textrangle}{}{Penalty parameter influencing the discontinuity of the solution (for each substance). Its default value 1 is sufficient in most cases. Higher value diminishes the inter-element jumps. $[-]$}
\KeyItem{\hyperB{SoluteTransport-DG-Data::bc-type}{bc\_type}}{abstract type: \Alink{IT::Field:R3 -> Enum[n]}{Field:R3 -> Enum[n]}}{\textlangle{\it optional }\textrangle}{}{Boundary condition type, possible values: inflow, dirichlet, neumann, robin. $[-]$}
\KeyItem{\hyperB{SoluteTransport-DG-Data::bc-flux}{bc\_flux}}{abstract type: \Alink{IT::Field:R3 -> Real[n]}{Field:R3 -> Real[n]}}{\textlangle{\it optional }\textrangle}{}{Flux in Neumann boundary condition. $[m^{1-d}kgs^{-1}]$}
\KeyItem{\hyperB{SoluteTransport-DG-Data::bc-robin-sigma}{bc\_robin\_sigma}}{abstract type: \Alink{IT::Field:R3 -> Real[n]}{Field:R3 -> Real[n]}}{\textlangle{\it optional }\textrangle}{}{Conductivity coefficient in Robin boundary condition. $[m^{4-d}s^{-1}]$}
\end{RecordType}

\begin{AbstractType}{\hyperB{IT::Field:R3 -> Enum[n]}{Field:R3 -> Enum[n]}}{}{}{Abstract record for all time-space functions.}
\Descendant{\Alink{IT::FieldConstant}{FieldConstant}}
\Descendant{\Alink{IT::FieldFormula}{FieldFormula}}
\Descendant{\Alink{IT::FieldPython}{FieldPython}}
\Descendant{\Alink{IT::FieldInterpolatedP0}{FieldInterpolatedP0}}
\Descendant{\Alink{IT::FieldElementwise}{FieldElementwise}}
\end{AbstractType}

\begin{RecordType}{\hyperB{IT::FieldConstant}{FieldConstant}}{\Alink{IT::Field:R3 -> Enum[n]}{Field:R3 -> Enum[n]}}{}{}{R3 -> Enum[n] Field constant in space.}
\KeyItem{\hyperB{FieldConstant::TYPE}{TYPE}}{selection: Field:R3 -> Enum[n]\_TYPE\_selection}{FieldConstant}{}{Sub-record selection.}
\KeyItem{\hyperB{FieldConstant::value}{value}}{String (generic)}{Array of HOW_TO_DETERMINE_TYPE}{\textlangle{\it obligatory }\textrangle}{}{Value of the constant field.\\For vector values, you can use scalar value to enter constant vector.\\For square NxN-matrix values, you can use:\\* vector of size N to enter diagonal matrix\\* vector of size (N+1)*N/2 to enter symmetric matrix (upper triangle, row by row)\\* scalar to enter multiple of the unit matrix.}
\end{RecordType}

\begin{SelectionType}{\hyperB{IT::TransportDG-BC-Type}{TransportDG\_BC\_Type}}{Types of boundary condition supported by the transport DG model (solute transport or heat transfer).}
\KeyItem{none}{Types of boundary condition supported by the transport DG model (solute transport or heat transfer).}
\KeyItem{dirichlet}{Types of boundary condition supported by the transport DG model (solute transport or heat transfer).}
\KeyItem{neumann}{Types of boundary condition supported by the transport DG model (solute transport or heat transfer).}
\KeyItem{robin}{Types of boundary condition supported by the transport DG model (solute transport or heat transfer).}
\KeyItem{inflow}{Types of boundary condition supported by the transport DG model (solute transport or heat transfer).}
\end{SelectionType}

\begin{SelectionType}{\hyperB{IT::DG-variant}{DG\_variant}}{Type of penalty term.}
\KeyItem{non-symmetric}{Type of penalty term.}
\KeyItem{incomplete}{Type of penalty term.}
\KeyItem{symmetric}{Type of penalty term.}
\end{SelectionType}

\begin{SelectionType}{\hyperB{IT::SoluteTransport-DG-Output}{SoluteTransport\_DG\_Output}}{Output record for DG solver for solute transport.}
\KeyItem{porosity}{Output record for DG solver for solute transport.}
\KeyItem{sources_density}{Output record for DG solver for solute transport.}
\KeyItem{sources_sigma}{Output record for DG solver for solute transport.}
\KeyItem{sources_conc}{Output record for DG solver for solute transport.}
\KeyItem{init_conc}{Output record for DG solver for solute transport.}
\KeyItem{disp_l}{Output record for DG solver for solute transport.}
\KeyItem{disp_t}{Output record for DG solver for solute transport.}
\KeyItem{diff_m}{Output record for DG solver for solute transport.}
\KeyItem{conc}{Output record for DG solver for solute transport.}
\KeyItem{fracture_sigma}{Output record for DG solver for solute transport.}
\KeyItem{dg_penalty}{Output record for DG solver for solute transport.}
\KeyItem{region_id}{Output record for DG solver for solute transport.}
\end{SelectionType}

\begin{RecordType}{\hyperB{IT::HeatTransfer-DG}{HeatTransfer\_DG}}{\Alink{IT::Transport}{Transport}}{}{}{DG solver for heat transfer.}
\KeyItem{\hyperB{HeatTransfer-DG::TYPE}{TYPE}}{selection: Transport\_TYPE\_selection}{HeatTransfer_DG}{}{Sub-record selection.}
\KeyItem{\hyperB{HeatTransfer-DG::time}{time}}{record: \Alink{IT::TimeGovernor}{TimeGovernor}}{\textlangle{\it obligatory }\textrangle}{}{Time governor setting for the secondary equation.}
\KeyItem{\hyperB{HeatTransfer-DG::balance}{balance}}{record: \Alink{IT::Balance}{Balance}}{\textlangle{\it obligatory }\textrangle}{}{Settings for computing balance.}
\KeyItem{\hyperB{HeatTransfer-DG::output-stream}{output\_stream}}{record: \Alink{IT::OutputStream}{OutputStream}}{\textlangle{\it obligatory }\textrangle}{}{Parameters of output stream.}
\KeyItem{\hyperB{HeatTransfer-DG::solver}{solver}}{record: \Alink{IT::Petsc}{Petsc}}{\textlangle{\it obligatory }\textrangle}{}{Linear solver for MH problem.}
\KeyItem{\hyperB{HeatTransfer-DG::input-fields}{input\_fields}}{String (generic)}{Array of HOW_TO_DETERMINE_TYPE}{\textlangle{\it obligatory }\textrangle}{}{}
\KeyItem{\hyperB{HeatTransfer-DG::dg-variant}{dg\_variant}}{selection: DG\_variant}{non-symmetric}{}{Variant of interior penalty discontinuous Galerkin method.}
\KeyItem{\hyperB{HeatTransfer-DG::dg-order}{dg\_order}}{String (generic)}{Integer<NumberRange {0 , 3}>}{\textlangle{\it value at declaration }\textrangle}{}{Polynomial order for finite element in DG method (order 0 is suitable if there is no diffusion/dispersion).}
\KeyItem{\hyperB{HeatTransfer-DG::output-fields}{output\_fields}}{String (generic)}{Array of HOW_TO_DETERMINE_TYPE}{\textlangle{\it value at declaration }\textrangle}{}{List of fields to write to output file.}
\end{RecordType}

\begin{RecordType}{\hyperB{IT::HeatTransfer-DG-Data}{HeatTransfer\_DG\_Data}}{}{}{Record to set fields of the equation.\\The fields are set only on the domain specified by one of the keys: 'region', 'rid', 'r_set'\\and after the time given by the key 'time'. The field setting can be overridden by\\ any HeatTransfer_DG_Data record that comes later in the boundary data array.}
\KeyItem{\hyperB{HeatTransfer-DG-Data::r-set}{r\_set}}{String (generic)}{\textlangle{\it optional }\textrangle}{}{Name of region set where to set fields.}
\KeyItem{\hyperB{HeatTransfer-DG-Data::region}{region}}{String (generic)}{\textlangle{\it optional }\textrangle}{}{Label of the region where to set fields.}
\KeyItem{\hyperB{HeatTransfer-DG-Data::rid}{rid}}{String (generic)}{Integer<NumberRange {0 , 2147483647}>}{\textlangle{\it optional }\textrangle}{}{ID of the region where to set fields.}
\KeyItem{\hyperB{HeatTransfer-DG-Data::time}{time}}{String (generic)}{Double<NumberRange {0 , 1.79769e+308}>}{\textlangle{\it value at declaration }\textrangle}{}{Apply field setting in this record after this time.\\These times have to form an increasing sequence.}
\KeyItem{\hyperB{HeatTransfer-DG-Data::bc-temperature}{bc\_temperature}}{abstract type: \Alink{IT::Field:R3 -> Real}{Field:R3 -> Real}}{\textlangle{\it optional }\textrangle}{}{Boundary value of temperature. $[K]$}
\KeyItem{\hyperB{HeatTransfer-DG-Data::init-temperature}{init\_temperature}}{abstract type: \Alink{IT::Field:R3 -> Real}{Field:R3 -> Real}}{\textlangle{\it optional }\textrangle}{}{Initial temperature. $[K]$}
\KeyItem{\hyperB{HeatTransfer-DG-Data::porosity}{porosity}}{abstract type: \Alink{IT::Field:R3 -> Real}{Field:R3 -> Real}}{\textlangle{\it optional }\textrangle}{}{Porosity. $[-]$}
\KeyItem{\hyperB{HeatTransfer-DG-Data::fluid-density}{fluid\_density}}{abstract type: \Alink{IT::Field:R3 -> Real}{Field:R3 -> Real}}{\textlangle{\it optional }\textrangle}{}{Density of fluid. $[m^{-3}kg]$}
\KeyItem{\hyperB{HeatTransfer-DG-Data::fluid-heat-capacity}{fluid\_heat\_capacity}}{abstract type: \Alink{IT::Field:R3 -> Real}{Field:R3 -> Real}}{\textlangle{\it optional }\textrangle}{}{Heat capacity of fluid. $[m^{2}s^{-2}K^{-1}]$}
\KeyItem{\hyperB{HeatTransfer-DG-Data::fluid-heat-conductivity}{fluid\_heat\_conductivity}}{abstract type: \Alink{IT::Field:R3 -> Real}{Field:R3 -> Real}}{\textlangle{\it optional }\textrangle}{}{Heat conductivity of fluid. $[mkgs^{-3}K^{-1}]$}
\KeyItem{\hyperB{HeatTransfer-DG-Data::solid-density}{solid\_density}}{abstract type: \Alink{IT::Field:R3 -> Real}{Field:R3 -> Real}}{\textlangle{\it optional }\textrangle}{}{Density of solid (rock). $[m^{-3}kg]$}
\KeyItem{\hyperB{HeatTransfer-DG-Data::solid-heat-capacity}{solid\_heat\_capacity}}{abstract type: \Alink{IT::Field:R3 -> Real}{Field:R3 -> Real}}{\textlangle{\it optional }\textrangle}{}{Heat capacity of solid (rock). $[m^{2}s^{-2}K^{-1}]$}
\KeyItem{\hyperB{HeatTransfer-DG-Data::solid-heat-conductivity}{solid\_heat\_conductivity}}{abstract type: \Alink{IT::Field:R3 -> Real}{Field:R3 -> Real}}{\textlangle{\it optional }\textrangle}{}{Heat conductivity of solid (rock). $[mkgs^{-3}K^{-1}]$}
\KeyItem{\hyperB{HeatTransfer-DG-Data::disp-l}{disp\_l}}{abstract type: \Alink{IT::Field:R3 -> Real}{Field:R3 -> Real}}{\textlangle{\it optional }\textrangle}{}{Longitudal heat dispersivity in fluid. $[m]$}
\KeyItem{\hyperB{HeatTransfer-DG-Data::disp-t}{disp\_t}}{abstract type: \Alink{IT::Field:R3 -> Real}{Field:R3 -> Real}}{\textlangle{\it optional }\textrangle}{}{Transversal heat dispersivity in fluid. $[m]$}
\KeyItem{\hyperB{HeatTransfer-DG-Data::fluid-thermal-source}{fluid\_thermal\_source}}{abstract type: \Alink{IT::Field:R3 -> Real}{Field:R3 -> Real}}{\textlangle{\it optional }\textrangle}{}{Thermal source density in fluid. $[m^{-1}kgs^{-3}]$}
\KeyItem{\hyperB{HeatTransfer-DG-Data::solid-thermal-source}{solid\_thermal\_source}}{abstract type: \Alink{IT::Field:R3 -> Real}{Field:R3 -> Real}}{\textlangle{\it optional }\textrangle}{}{Thermal source density in solid. $[m^{-1}kgs^{-3}]$}
\KeyItem{\hyperB{HeatTransfer-DG-Data::fluid-heat-exchange-rate}{fluid\_heat\_exchange\_rate}}{abstract type: \Alink{IT::Field:R3 -> Real}{Field:R3 -> Real}}{\textlangle{\it optional }\textrangle}{}{Heat exchange rate in fluid. $[s^{-1}]$}
\KeyItem{\hyperB{HeatTransfer-DG-Data::solid-heat-exchange-rate}{solid\_heat\_exchange\_rate}}{abstract type: \Alink{IT::Field:R3 -> Real}{Field:R3 -> Real}}{\textlangle{\it optional }\textrangle}{}{Heat exchange rate of source in solid. $[s^{-1}]$}
\KeyItem{\hyperB{HeatTransfer-DG-Data::fluid-ref-temperature}{fluid\_ref\_temperature}}{abstract type: \Alink{IT::Field:R3 -> Real}{Field:R3 -> Real}}{\textlangle{\it optional }\textrangle}{}{Reference temperature of source in fluid. $[K]$}
\KeyItem{\hyperB{HeatTransfer-DG-Data::solid-ref-temperature}{solid\_ref\_temperature}}{abstract type: \Alink{IT::Field:R3 -> Real}{Field:R3 -> Real}}{\textlangle{\it optional }\textrangle}{}{Reference temperature in solid. $[K]$}
\KeyItem{\hyperB{HeatTransfer-DG-Data::fracture-sigma}{fracture\_sigma}}{abstract type: \Alink{IT::Field:R3 -> Real[n]}{Field:R3 -> Real[n]}}{\textlangle{\it optional }\textrangle}{}{Coefficient of diffusive transfer through fractures (for each substance). $[-]$}
\KeyItem{\hyperB{HeatTransfer-DG-Data::dg-penalty}{dg\_penalty}}{abstract type: \Alink{IT::Field:R3 -> Real[n]}{Field:R3 -> Real[n]}}{\textlangle{\it optional }\textrangle}{}{Penalty parameter influencing the discontinuity of the solution (for each substance). Its default value 1 is sufficient in most cases. Higher value diminishes the inter-element jumps. $[-]$}
\KeyItem{\hyperB{HeatTransfer-DG-Data::bc-type}{bc\_type}}{abstract type: \Alink{IT::Field:R3 -> Enum[n]}{Field:R3 -> Enum[n]}}{\textlangle{\it optional }\textrangle}{}{Boundary condition type, possible values: inflow, dirichlet, neumann, robin. $[-]$}
\KeyItem{\hyperB{HeatTransfer-DG-Data::bc-flux}{bc\_flux}}{abstract type: \Alink{IT::Field:R3 -> Real[n]}{Field:R3 -> Real[n]}}{\textlangle{\it optional }\textrangle}{}{Flux in Neumann boundary condition. $[m^{1-d}kgs^{-1}]$}
\KeyItem{\hyperB{HeatTransfer-DG-Data::bc-robin-sigma}{bc\_robin\_sigma}}{abstract type: \Alink{IT::Field:R3 -> Real[n]}{Field:R3 -> Real[n]}}{\textlangle{\it optional }\textrangle}{}{Conductivity coefficient in Robin boundary condition. $[m^{4-d}s^{-1}]$}
\end{RecordType}

\begin{AbstractType}{\hyperB{IT::Field:R3 -> Enum[n]}{Field:R3 -> Enum[n]}}{}{}{Abstract record for all time-space functions.}
\Descendant{\Alink{IT::FieldConstant}{FieldConstant}}
\Descendant{\Alink{IT::FieldFormula}{FieldFormula}}
\Descendant{\Alink{IT::FieldPython}{FieldPython}}
\Descendant{\Alink{IT::FieldInterpolatedP0}{FieldInterpolatedP0}}
\Descendant{\Alink{IT::FieldElementwise}{FieldElementwise}}
\end{AbstractType}

\begin{SelectionType}{\hyperB{IT::HeatTransfer-DG-Output}{HeatTransfer\_DG\_Output}}{Selection for output fields of DG solver for heat transfer.}
\KeyItem{init_temperature}{Selection for output fields of DG solver for heat transfer.}
\KeyItem{porosity}{Selection for output fields of DG solver for heat transfer.}
\KeyItem{fluid_density}{Selection for output fields of DG solver for heat transfer.}
\KeyItem{fluid_heat_capacity}{Selection for output fields of DG solver for heat transfer.}
\KeyItem{fluid_heat_conductivity}{Selection for output fields of DG solver for heat transfer.}
\KeyItem{solid_density}{Selection for output fields of DG solver for heat transfer.}
\KeyItem{solid_heat_capacity}{Selection for output fields of DG solver for heat transfer.}
\KeyItem{solid_heat_conductivity}{Selection for output fields of DG solver for heat transfer.}
\KeyItem{disp_l}{Selection for output fields of DG solver for heat transfer.}
\KeyItem{disp_t}{Selection for output fields of DG solver for heat transfer.}
\KeyItem{fluid_thermal_source}{Selection for output fields of DG solver for heat transfer.}
\KeyItem{solid_thermal_source}{Selection for output fields of DG solver for heat transfer.}
\KeyItem{fluid_heat_exchange_rate}{Selection for output fields of DG solver for heat transfer.}
\KeyItem{solid_heat_exchange_rate}{Selection for output fields of DG solver for heat transfer.}
\KeyItem{fluid_ref_temperature}{Selection for output fields of DG solver for heat transfer.}
\KeyItem{solid_ref_temperature}{Selection for output fields of DG solver for heat transfer.}
\KeyItem{temperature}{Selection for output fields of DG solver for heat transfer.}
\KeyItem{fracture_sigma}{Selection for output fields of DG solver for heat transfer.}
\KeyItem{dg_penalty}{Selection for output fields of DG solver for heat transfer.}
\KeyItem{region_id}{Selection for output fields of DG solver for heat transfer.}
\end{SelectionType}

